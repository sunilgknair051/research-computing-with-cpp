\section{Lecture 1: C++ for Research}\label{lecture-1-c-for-research}

\subsection{Course Overview}\label{course-overview}

\subsubsection{Part 1}\label{part-1}

\begin{itemize}
\itemsep1pt\parskip0pt\parsep0pt
\item
  Using C++ in research
\item
  Better C++

  \begin{itemize}
  \itemsep1pt\parskip0pt\parsep0pt
  \item
    Reliable
  \item
    Reproducible
  \item
    Good science
  \item
    Libraries
  \end{itemize}
\end{itemize}

\subsubsection{Part 2}\label{part-2}

\begin{itemize}
\itemsep1pt\parskip0pt\parsep0pt
\item
  HPC concepts
\item
  Shared memory parallelism - \href{http://www.openmp.org}{OpenMP}
\item
  Distributed memory parallelism - \href{http://www.open-mpi.org}{MPI}
\end{itemize}

\subsubsection{Course Aims}\label{course-aims}

\begin{itemize}
\itemsep1pt\parskip0pt\parsep0pt
\item
  Teach how to do research with C++
\item
  Optimise your research output
\item
  A taster for various technologies
\item
  Not just C++ syntax, Google/Compiler could tell you that!
\end{itemize}

\subsubsection{Pre-requisites}\label{pre-requisites}

\begin{itemize}
\itemsep1pt\parskip0pt\parsep0pt
\item
  Use of command line (Unix) shell
\item
  You are already doing some C++
\item
  You are familiar with your compiler
\item
  (Maybe) You are happy with the concept of classes
\item
  (Maybe) You know C++ up to templates?
\item
  You are familiar with development eg. version control

  \begin{itemize}
  \itemsep1pt\parskip0pt\parsep0pt
  \item
    Git: \url{https://git-scm.com/}
  \end{itemize}
\end{itemize}

\subsubsection{Course Notes}\label{course-notes}

\begin{itemize}
\itemsep1pt\parskip0pt\parsep0pt
\item
  Revise some software Engineering:
  \href{http://github-pages.ucl.ac.uk/rsd-engineeringcourse/}{MPHY0021}
\item
  Register with Moodle: \href{https://moodle.ucl.ac.uk/}{PHAS0100}

  \begin{itemize}
  \itemsep1pt\parskip0pt\parsep0pt
  \item
    contact lecturer for key to self-register
  \item
    guest key to look around is ``996135''
  \end{itemize}
\item
  Online notes:
  \href{http://rits.github-pages.ucl.ac.uk/research-computing-with-cpp/}{PHAS0100}
\end{itemize}

\subsubsection{Course Assessment}\label{course-assessment}

\begin{itemize}
\itemsep1pt\parskip0pt\parsep0pt
\item
  2 pieces coursework - 40 hours each

  \begin{itemize}
  \itemsep1pt\parskip0pt\parsep0pt
  \item
    See assessment section for details
  \end{itemize}
\end{itemize}

\subsubsection{Course Community}\label{course-community}

\begin{itemize}
\itemsep1pt\parskip0pt\parsep0pt
\item
  UCL Research Programming Hub:
  \href{http://research-programming.ucl.ac.uk/}{http://research-programming.ucl.ac.uk}
\item
  Slack:
  \href{https://ucl-programming-hub.slack.com/}{https://ucl-programming-hub.slack.com}
\end{itemize}

\subsection{Lecture 1: C++ In Research}\label{lecture-1-c-in-research}

\subsubsection{Problems In Research}\label{problems-in-research}

\begin{itemize}
\itemsep1pt\parskip0pt\parsep0pt
\item
  Poor quality software
\item
  Excuses

  \begin{itemize}
  \itemsep1pt\parskip0pt\parsep0pt
  \item
    I'm not a software engineer
  \item
    I don't have time
  \item
    It's just a prototype
  \item
    I'm unsure of my code (scared to share)
  \end{itemize}
\end{itemize}

\subsubsection{C++ Disadvantages}\label{c-disadvantages}

Some people say:

\begin{itemize}
\itemsep1pt\parskip0pt\parsep0pt
\item
  Compiled language

  \begin{itemize}
  \itemsep1pt\parskip0pt\parsep0pt
  \item
    (compiler versions, libraries, platform specific etc)
  \end{itemize}
\item
  Perceived as difficult, error prone, wordy, unfriendly syntax
\item
  Result: It's more trouble than its worth?
\end{itemize}

\subsubsection{C++ Advantages}\label{c-advantages}

\begin{itemize}
\itemsep1pt\parskip0pt\parsep0pt
\item
  Fast, code compiled to machine code
\item
  Stable, evolving standard, powerful notation, improving
\item
  Lots of libraries, Boost, Qt, VTK, ITK etc.
\item
  Nice integration with CUDA, OpenACC, OpenCL, OpenMP, OpenMPI
\item
  Result: Good reasons to use it, or you may \emph{have} to use it
\end{itemize}

\subsubsection{Research Programming}\label{research-programming}

\begin{itemize}
\itemsep1pt\parskip0pt\parsep0pt
\item
  Software is always expensive

  \begin{itemize}
  \itemsep1pt\parskip0pt\parsep0pt
  \item
    Famous Book:
    \href{http://www.amazon.co.uk/Mythical-Man-month-Essays-Software-Engineering/dp/0201835959/ref=sr_1_1?ie=UTF8\&qid=1452507457\&sr=8-1\&keywords=mythical+man+month}{Mythical
    Man Month}
  \end{itemize}
\item
  Research programming is different to product development:

  \begin{itemize}
  \itemsep1pt\parskip0pt\parsep0pt
  \item
    What is the end product?
  \end{itemize}
\end{itemize}

\subsubsection{Development Methodology?}\label{development-methodology}

\begin{itemize}
\itemsep1pt\parskip0pt\parsep0pt
\item
  Will software engineering methods help?

  \begin{itemize}
  \itemsep1pt\parskip0pt\parsep0pt
  \item
    \href{https://en.wikipedia.org/wiki/Waterfall_model}{Waterfall}
  \item
    \href{https://en.wikipedia.org/wiki/Agile_software_development}{Agile}
  \end{itemize}
\item
  At the `concept discovery' stage, probably too early to talk about
  product development
\end{itemize}

\subsubsection{Approach}\label{approach}

\begin{itemize}
\itemsep1pt\parskip0pt\parsep0pt
\item
  What am I trying to achieve?
\item
  How do I maximise my research output?
\item
  What is the best pathway to success?
\item
  How do I de-risk (get results, meet deadlines) my research?
\item
  Software is an important part of scientific reproducibility,
  authorship, credibility.
\end{itemize}

\subsubsection{1. Types of Code}\label{types-of-code}

\begin{itemize}
\itemsep1pt\parskip0pt\parsep0pt
\item
  What are you trying to achieve?
\item
  Divide code:

  \begin{itemize}
  \itemsep1pt\parskip0pt\parsep0pt
  \item
    Your algorithm
  \item
    Testing code
  \item
    Data analysis
  \item
    User Interface
  \item
    Glue code
  \item
    Deployment code
  \item
    Scientific output (a paper)
  \end{itemize}
\end{itemize}

Question: What type of code is C++ good for? Question: Should all code
be in C++?

\subsubsection{2. Maximise Your Value}\label{maximise-your-value}

\begin{itemize}
\itemsep1pt\parskip0pt\parsep0pt
\item
  Developer time is expensive
\item
  Your brain is your asset
\item
  Write as little code as possible
\item
  Focus tightly on your hypothesis
\item
  Write the minimum code that produces a paper
\end{itemize}

Don't fall into the trap ``Hey, I'll write a framework for that''

\subsubsection{3. Ask Advice}\label{ask-advice}

\begin{itemize}
\itemsep1pt\parskip0pt\parsep0pt
\item
  Before contemplating a new piece of software

  \begin{itemize}
  \itemsep1pt\parskip0pt\parsep0pt
  \item
    Ask advice - \href{https://ucl-programming-hub.slack.com/}{Slack
    Channel}
  \item
    Review libraries and use them.
  \item
    Check libraries are suitable, and sustainable.
  \item
    Read
    \href{http://development.rc.ucl.ac.uk/training/engineering/ch04packaging/01Libraries.html}{Libraries}
    section from
    \href{http://github-pages.ucl.ac.uk/rsd-engineeringcourse/}{Software
    Engineering} course
  \item
    Ask about best practices
  \end{itemize}
\end{itemize}

\subsubsection{Debunking The Excuses}\label{debunking-the-excuses}

\begin{itemize}
\itemsep1pt\parskip0pt\parsep0pt
\item
  I'm not a software engineer

  \begin{itemize}
  \itemsep1pt\parskip0pt\parsep0pt
  \item
    Learn effective, minimal tools
  \end{itemize}
\item
  I don't have time

  \begin{itemize}
  \itemsep1pt\parskip0pt\parsep0pt
  \item
    Unit testing to save time
  \item
    Choose your battles/languages wisely
  \end{itemize}
\item
  I'm unsure of my code

  \begin{itemize}
  \itemsep1pt\parskip0pt\parsep0pt
  \item
    Share, collaborate
  \end{itemize}
\end{itemize}

\subsubsection{What Isn't This Course?}\label{what-isnt-this-course}

We are NOT suggesting that:

\begin{itemize}
\itemsep1pt\parskip0pt\parsep0pt
\item
  C++ is the solution to all problems.
\item
  You should write all parts of your code in C++.
\end{itemize}

\subsubsection{What Is This Course?}\label{what-is-this-course}

We aim to:

\begin{itemize}
\itemsep1pt\parskip0pt\parsep0pt
\item
  Improve your C++ (and associated technologies).
\item
  Introduction to High Performance Computing (HPC).
\end{itemize}

So that:

\begin{itemize}
\itemsep1pt\parskip0pt\parsep0pt
\item
  Apply it to research in a pragmatic fashion.
\item
  You use the right tool for the job.
\end{itemize}

\subsection{Git}\label{git}

\subsubsection{Git Introduction}\label{git-introduction}

\begin{itemize}
\itemsep1pt\parskip0pt\parsep0pt
\item
  This is a practical course
\item
  We will use git for version control
\item
  Submit git repository for coursework
\item
  Here we provide a very minimal introduction
\end{itemize}

\subsubsection{Git Resources}\label{git-resources}

\begin{itemize}
\itemsep1pt\parskip0pt\parsep0pt
\item
  Complete beginner - \href{https://try.github.io}{Try Git}
\item
  \href{https://git-scm.com/book/en/v2}{Git book by Scott Chacon}
\item
  \href{http://github-pages.ucl.ac.uk/rsd-engineeringcourse/ch02git/}{Git
  section} of MPHYG001
\item
  \href{https://github.com/UCL/rsd-engineeringcourse}{MPHYG001 repo}
\end{itemize}

\subsubsection{Git Walk Through - 1}\label{git-walk-through---1}

(demo on command line)

\begin{itemize}
\itemsep1pt\parskip0pt\parsep0pt
\item
  Creating your own repo:

  \begin{itemize}
  \itemsep1pt\parskip0pt\parsep0pt
  \item
    git init
  \item
    git add
  \item
    git commit
  \item
    git log
  \item
    git status
  \item
    git remote add
  \item
    git push
  \end{itemize}
\end{itemize}

\subsubsection{Git Walk Through - 2}\label{git-walk-through---2}

(demo on command line)

\begin{itemize}
\itemsep1pt\parskip0pt\parsep0pt
\item
  Working on existing repo:

  \begin{itemize}
  \itemsep1pt\parskip0pt\parsep0pt
  \item
    git clone
  \item
    git add
  \item
    git commit
  \item
    git log
  \item
    git status\\
  \item
    git push
  \item
    git pull
  \end{itemize}
\end{itemize}

\subsubsection{Git Walk Through - 3}\label{git-walk-through---3}

\begin{itemize}
\itemsep1pt\parskip0pt\parsep0pt
\item
  Cloning or forking:

  \begin{itemize}
  \itemsep1pt\parskip0pt\parsep0pt
  \item
    If you have write permission to a repo: clone it, and make
    modifications
  \item
    If you don't: fork it, to make your own version, then clone that and
    make modifications.
  \end{itemize}
\end{itemize}

\subsubsection{Homework - 1}\label{homework---1}

\begin{itemize}
\itemsep1pt\parskip0pt\parsep0pt
\item
  Register Github
\item
  Register for private repos - free for academia.
\item
  Find project of interest - try cloning it, make edits, can you push?
\item
  Find project of interest - try forking it, make edits, can you push?
\end{itemize}

\subsection{CMake}\label{cmake}

\subsubsection{Ever worked in Industry?}\label{ever-worked-in-industry}

\begin{itemize}
\itemsep1pt\parskip0pt\parsep0pt
\item
  (0-3yrs) Junior Developer - given environment, team support
\item
  (4-6yrs) Senior Developer - given environment, leading team
\item
  (7+ years) Architect - chose tools, environment, design code
\item
  Only cross-platform if product/business demands it
\item
  All developers told to use the given platform no choice
\end{itemize}

\subsubsection{Ever worked in Research?}\label{ever-worked-in-research}

\begin{itemize}
\itemsep1pt\parskip0pt\parsep0pt
\item
  All prototyping, no scope
\item
  Start from scratch, little support
\item
  No end product, no nice examples
\item
  Cutting edge use of maths/science/technology
\item
  Share with others on other platforms
\item
  Develop on Windows, run on cluster (Linux)
\end{itemize}

\subsubsection{Research Software Engineering
Dilemma}\label{research-software-engineering-dilemma}

\begin{itemize}
\itemsep1pt\parskip0pt\parsep0pt
\item
  Comparing Research with Industry, in Research you have:

  \begin{itemize}
  \itemsep1pt\parskip0pt\parsep0pt
  \item
    Least experienced developers
  \item
    with the least support
  \item
    developing cross-platform
  \item
    No clear specification or scope
  \end{itemize}
\item
  Struggle of C++ is often not the language its the environment
\end{itemize}

\subsubsection{Build Environment}\label{build-environment}

\begin{itemize}
\itemsep1pt\parskip0pt\parsep0pt
\item
  Windows: Visual Studio solution files
\item
  Linux: Makefiles
\item
  Mac: XCode projects / Makefiles
\end{itemize}

Question: How was your last project built?

\subsubsection{CMake Introduction}\label{cmake-introduction}

\begin{itemize}
\itemsep1pt\parskip0pt\parsep0pt
\item
  This is a practical course
\item
  We will use CMake as a build tool
\item
  CMake produces

  \begin{itemize}
  \itemsep1pt\parskip0pt\parsep0pt
  \item
    Windows: Visual Studio project files
  \item
    Linux: Make files
  \item
    Mac: XCode projects, Make files
  \end{itemize}
\item
  So you write 1 build language (CMake) and run on multi-platform.
\item
  This course will provide most CMake code and boiler plate code for
  you, so we can focus more on C++. But you are expected to google CMake
  issues and work with CMake.
\end{itemize}

\subsubsection{CMake Usage Linux/Mac}\label{cmake-usage-linuxmac}

Demo an ``out-of-source'' build

\begin{Shaded}
\begin{Highlighting}[]
\KeywordTok{mkdir} \NormalTok{cmake_demo}
\KeywordTok{cd} \NormalTok{cmake_demo}
\KeywordTok{git} \NormalTok{clone https://github.com/MattClarkson/CMakeHelloWorld}
\KeywordTok{cd} \NormalTok{CMakeHelloWorld}
\KeywordTok{mkdir} \NormalTok{build}
\KeywordTok{cd} \NormalTok{build}
\KeywordTok{cmake} \NormalTok{..}
\KeywordTok{make}
\end{Highlighting}
\end{Shaded}

\subsubsection{Homework - 2}\label{homework---2}

\begin{itemize}
\itemsep1pt\parskip0pt\parsep0pt
\item
  Build https://github.com/MattClarkson/CMakeHelloWorld.git
\item
  Ensure you do ``out-of-source'' builds
\item
  Use CMake to configure separate Debug and Release versions
\item
  Add code to hello.cpp:

  \begin{itemize}
  \itemsep1pt\parskip0pt\parsep0pt
  \item
    On Linux/Mac re-compile just using make
  \end{itemize}
\end{itemize}

\subsubsection{Homework - 3}\label{homework---3}

\begin{itemize}
\itemsep1pt\parskip0pt\parsep0pt
\item
  Build https://github.com/MattClarkson/CMakeHelloWorld.git
\item
  Exit all code editors
\item
  Rename hello.cpp
\item
  Change CMakeLists.txt accordingly
\item
  Notice: The executable name and .cpp file name can be different
\item
  In your build folder, just try rebuilding.
\item
  You should see that CMake is re-triggered, so you get a cmake/compile
  cycle.
\end{itemize}

\subsection{CMake Basics}\label{cmake-basics}

\subsubsection{Compiling Basics}\label{compiling-basics}

Question: How does a compiler work?

\subsubsection{How does a Compiler
Work?}\label{how-does-a-compiler-work}

Question: How does a compiler work?

\begin{itemize}
\itemsep1pt\parskip0pt\parsep0pt
\item
  (Don't quote any of this in a compiler theory course!)
\item
  Preprocessing .cpp/.cxx into pure source files
\item
  Source files compiled, one by one into .o/.obj
\item
  executable compiled to .o/.obj
\item
  executable linked against all .o, and all libraries
\end{itemize}

That's what you are trying to describe in CMake.

\subsubsection{CMake - Directory
Structure}\label{cmake---directory-structure}

\begin{itemize}
\itemsep1pt\parskip0pt\parsep0pt
\item
  CMake starts with top-level CMakeLists.txt
\item
  CMakeLists.txt is read top-to-bottom
\item
  All CMake code goes in CMakeLists.txt or files included from a
  CMakeLists.txt
\item
  You can sub-divide your code into separate folders.
\item
  If you \texttt{add\_subdirectory}, CMake will go into that directory
  and start to process the CMakeLists.txt therein. Once finished it will
  exit, go back to directory above and continue where it left off.
\item
  e.g.~top level CMakeLists.txt
\end{itemize}

\begin{Shaded}
\begin{Highlighting}[]
\KeywordTok{project}\NormalTok{(MYPROJECT }\OtherTok{VERSION} \NormalTok{0.0.0)}
\KeywordTok{add_subdirectory}\NormalTok{(Code)}
\KeywordTok{if}\NormalTok{(BUILD_TESTING)}
  \KeywordTok{set}\NormalTok{()}
  \KeywordTok{include_directories}\NormalTok{()}
  \KeywordTok{add_subdirectory}\NormalTok{(Testing)}
\KeywordTok{endif}\NormalTok{()}
\end{Highlighting}
\end{Shaded}

\subsubsection{CMake - Define Targets}\label{cmake---define-targets}

\begin{itemize}
\itemsep1pt\parskip0pt\parsep0pt
\item
  Describe a target, e.g.~Library, Application, Plugin
\end{itemize}

\begin{Shaded}
\begin{Highlighting}[]
\KeywordTok{add_executable}\NormalTok{(hello hello.cpp)}
\end{Highlighting}
\end{Shaded}

\begin{itemize}
\itemsep1pt\parskip0pt\parsep0pt
\item
  Note: You don't write compile commands
\item
  You tell CMake what things need compiling to build a given target.
  CMake works out the compile commands!
\end{itemize}

\subsubsection{CMake - Order Dependent}\label{cmake---order-dependent}

\begin{itemize}
\itemsep1pt\parskip0pt\parsep0pt
\item
  You can't say ``build Y and link to X'' if X not defined
\item
  So, imagine in a larger project
\end{itemize}

\begin{Shaded}
\begin{Highlighting}[]
\KeywordTok{add_library}\NormalTok{(libA a.cpp b.cpp c.cpp)}
\KeywordTok{add_library}\NormalTok{(libZ x.cpp y.cpp z.cpp)}
\KeywordTok{target_link_libraries}\NormalTok{(libZ libA)}
\KeywordTok{add_executable}\NormalTok{(myAlgorithm algo.cpp) }\CommentTok{# contains main()}
\KeywordTok{target_link_libraries}\NormalTok{(myAlgorithm libA libZ }\DecValTok{$\{THIRD_PARTY_LIBS\}}\NormalTok{)}
\end{Highlighting}
\end{Shaded}

\begin{itemize}
\itemsep1pt\parskip0pt\parsep0pt
\item
  So, logically, its a big, ordered set of build commands.
\end{itemize}

\subsubsection{Homework - 4}\label{homework---4}

\begin{itemize}
\itemsep1pt\parskip0pt\parsep0pt
\item
  Build https://github.com/MattClarkson/CMakeLibraryAndApp.git
\item
  Look through .cpp/.h code. Ask questions if you don't understand it.
\item
  What is an ``include guard''?
\item
  What is a namespace?
\item
  Look at .travis.yml and appveyor.yml - cross platform testing, free
  for open-source
\item
  Look at myApp.cpp, does it make sense?
\item
  Look at CMakeLists.txt, does it make sense?
\item
  Look for examples on the web, e.g.
  \href{https://lorensen.github.io/VTKExamples/site/Cxx/GeometricObjects/Cone/}{VTK}
\end{itemize}

\subsection{Intermediate CMake}\label{intermediate-cmake}

\subsubsection{What's next?}\label{whats-next}

\begin{itemize}
\itemsep1pt\parskip0pt\parsep0pt
\item
  Most people learn CMake by pasting snippets from around the web
\item
  As project gets larger, its more complex
\item
  Researchers tend to just ``stick with what they have.''
\item
  i.e.~just keep piling more code into the same file.
\item
  Want to show you a reasonable template project.
\end{itemize}

\subsubsection{Homework - 5}\label{homework---5}

\begin{itemize}
\itemsep1pt\parskip0pt\parsep0pt
\item
  Build https://github.com/MattClarkson/CMakeCatch2.git
\item
  If open-source, use travis and appveyor from day 1.
\item
  We will go through top-level CMakeLists.txt in class.
\item
  See separate \texttt{Code} and \texttt{Testing} folders
\item
  Separate \texttt{Lib} and \texttt{CommandLineApps} and
  \texttt{3rdParty}
\item
  You should focus on

  \begin{itemize}
  \itemsep1pt\parskip0pt\parsep0pt
  \item
    Write a good library
  \item
    Unit test it
  \item
    Then it can be called from command line, wrapped in Python, used via
    GUI.
  \end{itemize}
\end{itemize}

\subsubsection{Homework - 6}\label{homework---6}

\begin{itemize}
\itemsep1pt\parskip0pt\parsep0pt
\item
  Try renaming stuff to create a library of your choice.
\item
  Create a github account, github repo, Appveyor and Travis account
\item
  Try to get your code running on 3 platforms
\item
  If you can, consider using this repo for your research
\item
  Discuss

  \begin{itemize}
  \itemsep1pt\parskip0pt\parsep0pt
  \item
    Debug / Release builds
  \item
    Static versus Dynamic
  \item
    declspec import/export
  \item
    Issues with running command line app? Windows/Linux/Mac
  \end{itemize}
\end{itemize}

\subsubsection{Looking forward}\label{looking-forward}

In the remainder of this course we cover

\begin{itemize}
\itemsep1pt\parskip0pt\parsep0pt
\item
  Some compiler options
\item
  Using libraries
\item
  Including libraries in CMake
\item
  Unit testing
\item
  i.e.~How to put together a C++ project
\item
  in addition to actual C++ and HPC
\end{itemize}

\subsection{Unit Testing}\label{unit-testing}

\subsubsection{What is Unit Testing?}\label{what-is-unit-testing}

At a high level

\begin{itemize}
\itemsep1pt\parskip0pt\parsep0pt
\item
  Way of testing code.
\item
  Unit

  \begin{itemize}
  \itemsep1pt\parskip0pt\parsep0pt
  \item
    Smallest `atomic' chunk of code
  \item
    i.e.~Function, could be a Class
  \end{itemize}
\item
  See also:

  \begin{itemize}
  \itemsep1pt\parskip0pt\parsep0pt
  \item
    Integration/System Testing
  \item
    Regression Testing
  \item
    User Acceptance Testing
  \end{itemize}
\end{itemize}

\subsubsection{Benefits of Unit
Testing?}\label{benefits-of-unit-testing}

\begin{itemize}
\itemsep1pt\parskip0pt\parsep0pt
\item
  Certainty of correctness
\item
  (Scientific Rigour)
\item
  Influences and improves design
\item
  Confidence to refactor, improve
\end{itemize}

\subsubsection{Drawbacks for Unit
Testing?}\label{drawbacks-for-unit-testing}

\begin{itemize}
\itemsep1pt\parskip0pt\parsep0pt
\item
  Don't know how

  \begin{itemize}
  \itemsep1pt\parskip0pt\parsep0pt
  \item
    This course will help
  \end{itemize}
\item
  Takes too much time

  \begin{itemize}
  \itemsep1pt\parskip0pt\parsep0pt
  \item
    Really?
  \item
    IT SAVES TIME in the long run
  \end{itemize}
\end{itemize}

\subsubsection{Unit Testing Frameworks}\label{unit-testing-frameworks}

Generally, all very similar

\begin{itemize}
\itemsep1pt\parskip0pt\parsep0pt
\item
  JUnit (Java), NUnit (.net?), CppUnit, phpUnit,
\item
  Basically

  \begin{itemize}
  \itemsep1pt\parskip0pt\parsep0pt
  \item
    Macros (C++), methods (Java) to test conditions
  \item
    Macros (C++), reflection (Java) to run/discover tests
  \item
    Ways of looking at results.

    \begin{itemize}
    \itemsep1pt\parskip0pt\parsep0pt
    \item
      Java/Eclipse: Integrated with IDE
    \item
      Log file or standard output
    \end{itemize}
  \end{itemize}
\end{itemize}

\subsection{Unit Testing Example}\label{unit-testing-example}

\subsubsection{How To Start}\label{how-to-start}

We discuss

\begin{itemize}
\itemsep1pt\parskip0pt\parsep0pt
\item
  Basic Example
\item
  Some tips
\end{itemize}

Then its down to the developer/artist.

\subsubsection{C++ Frameworks}\label{c-frameworks}

To Consider:

\begin{itemize}
\itemsep1pt\parskip0pt\parsep0pt
\item
  \href{https://github.com/philsquared/Catch}{Catch}
\item
  \href{https://code.google.com/p/googletest/}{GoogleTest}
\item
  \href{http://qt-project.org/doc/qt-4.8/qtestlib-manual.html}{QTestLib}
\item
  \href{http://www.boost.org/doc/libs/1_57_0/libs/test/doc/html/index.html}{BoostTest}
\item
  \href{http://cpptest.sourceforge.net/}{CppTest}
\item
  \href{http://sourceforge.net/projects/cppunit/}{CppUnit}
\end{itemize}

\subsubsection{Worked Example}\label{worked-example}

\begin{itemize}
\itemsep1pt\parskip0pt\parsep0pt
\item
  Borrowed from

  \begin{itemize}
  \itemsep1pt\parskip0pt\parsep0pt
  \item
    \href{https://github.com/philsquared/Catch/blob/master/docs/tutorial.md}{Catch
    Tutorial}
  \item
    and
    \href{https://code.google.com/p/googletest/wiki/V1_7_Primer}{Googletest
    Primer}
  \end{itemize}
\item
  We use \href{https://github.com/philsquared/Catch}{Catch}, so notes
  are compilable
\item
  But the concepts are the same
\end{itemize}

\subsubsection{Code}\label{code}

To keep it simple for now we do this in one file:

\begin{Shaded}
\begin{Highlighting}[]
\OtherTok{#define CATCH_CONFIG_MAIN  }\CommentTok{// This tells Catch to provide a main() - only do this in one cpp file}
\OtherTok{#include "../catch/catch.hpp"}

\DataTypeTok{unsigned} \DataTypeTok{int} \NormalTok{Factorial( }\DataTypeTok{unsigned} \DataTypeTok{int} \NormalTok{number ) \{}
    \KeywordTok{return} \NormalTok{number <= }\DecValTok{1} \NormalTok{? number : Factorial(number}\DecValTok{-1}\NormalTok{)*number;}
\NormalTok{\}}

\NormalTok{TEST_CASE( }\StringTok{"Factorials are computed"}\NormalTok{, }\StringTok{"[factorial]"} \NormalTok{) \{}
    \NormalTok{REQUIRE( Factorial(}\DecValTok{1}\NormalTok{) == }\DecValTok{1} \NormalTok{);}
    \NormalTok{REQUIRE( Factorial(}\DecValTok{2}\NormalTok{) == }\DecValTok{2} \NormalTok{);}
    \NormalTok{REQUIRE( Factorial(}\DecValTok{3}\NormalTok{) == }\DecValTok{6} \NormalTok{);}
    \NormalTok{REQUIRE( Factorial(}\DecValTok{10}\NormalTok{) == }\DecValTok{3628800} \NormalTok{);}
\NormalTok{\}}
\end{Highlighting}
\end{Shaded}

Produces this output when run:

\begin{verbatim}
===============================================================================
All tests passed (4 assertions in 1 test case)

\end{verbatim}

\subsubsection{Principles}\label{principles}

So, typically we have

\begin{itemize}
\itemsep1pt\parskip0pt\parsep0pt
\item
  Some \texttt{\#include} to get test framework
\item
  Our code that we want to test
\item
  Then make some assertions
\end{itemize}

\subsubsection{Catch / GoogleTest}\label{catch-googletest}

For example, in \href{https://github.com/philsquared/Catch}{Catch}:

\begin{Shaded}
\begin{Highlighting}[]
    \CommentTok{// TEST_CASE(<unique test name>, <test case name>)}
    \NormalTok{TEST_CASE( }\StringTok{"Factorials are computed"}\NormalTok{, }\StringTok{"[factorial]"} \NormalTok{) \{}
        \NormalTok{REQUIRE( Factorial(}\DecValTok{2}\NormalTok{) == }\DecValTok{2} \NormalTok{);}
        \NormalTok{REQUIRE( Factorial(}\DecValTok{3}\NormalTok{) == }\DecValTok{6} \NormalTok{);}
    \NormalTok{\}}
\end{Highlighting}
\end{Shaded}

In \href{https://code.google.com/p/googletest/}{GoogleTest}:

\begin{Shaded}
\begin{Highlighting}[]
    \CommentTok{// TEST(<test case name>, <unique test name>)}
    \NormalTok{TEST(FactorialTest, HandlesPositiveInput) \{}
      \NormalTok{EXPECT_EQ(}\DecValTok{2}\NormalTok{, Factorial(}\DecValTok{2}\NormalTok{));}
      \NormalTok{EXPECT_EQ(}\DecValTok{6}\NormalTok{, Factorial(}\DecValTok{3}\NormalTok{));}
    \NormalTok{\}}
\end{Highlighting}
\end{Shaded}

all done via C++ macros.

\subsubsection{Tests That Fail}\label{tests-that-fail}

What about Factorial of zero? Adding

\begin{Shaded}
\begin{Highlighting}[]
    \NormalTok{REQUIRE( Factorial(}\DecValTok{0}\NormalTok{) == }\DecValTok{1} \NormalTok{);}
\end{Highlighting}
\end{Shaded}

Produces something like:

\begin{Shaded}
\begin{Highlighting}[]
    \NormalTok{factorial2.cc:}\DecValTok{9}\NormalTok{: FAILED:}
    \NormalTok{REQUIRE( Factorial(}\DecValTok{0}\NormalTok{) == }\DecValTok{1} \NormalTok{)}
    \NormalTok{with expansion:}
    \DecValTok{0} \NormalTok{== }\DecValTok{1}
\end{Highlighting}
\end{Shaded}

\subsubsection{Fix the Failing Test}\label{fix-the-failing-test}

Leading to:

\begin{Shaded}
\begin{Highlighting}[]
\OtherTok{#define CATCH_CONFIG_MAIN  }\CommentTok{// This tells Catch to provide a main() - only do this in one cpp file}
\OtherTok{#include "../catch/catch.hpp"}

\DataTypeTok{unsigned} \DataTypeTok{int} \NormalTok{Factorial( }\DataTypeTok{unsigned} \DataTypeTok{int} \NormalTok{number ) \{}
    \CommentTok{//return number <= 1 ? number : Factorial(number-1)*number;}
    \KeywordTok{return} \NormalTok{number > }\DecValTok{1} \NormalTok{? Factorial(number}\DecValTok{-1}\NormalTok{)*number : }\DecValTok{1}\NormalTok{;}
\NormalTok{\}}

\NormalTok{TEST_CASE( }\StringTok{"Factorials are computed"}\NormalTok{, }\StringTok{"[factorial]"} \NormalTok{) \{}
    \NormalTok{REQUIRE( Factorial(}\DecValTok{0}\NormalTok{) == }\DecValTok{1} \NormalTok{);}
    \NormalTok{REQUIRE( Factorial(}\DecValTok{1}\NormalTok{) == }\DecValTok{1} \NormalTok{);}
    \NormalTok{REQUIRE( Factorial(}\DecValTok{2}\NormalTok{) == }\DecValTok{2} \NormalTok{);}
    \NormalTok{REQUIRE( Factorial(}\DecValTok{3}\NormalTok{) == }\DecValTok{6} \NormalTok{);}
    \NormalTok{REQUIRE( Factorial(}\DecValTok{10}\NormalTok{) == }\DecValTok{3628800} \NormalTok{);}
\NormalTok{\}}
\end{Highlighting}
\end{Shaded}

which passes:

\begin{verbatim}
===============================================================================
All tests passed (5 assertions in 1 test case)

\end{verbatim}

\subsubsection{Test Macros}\label{test-macros}

Each framework has a variety of macros to test for failure.
\href{https://github.com/philsquared/Catch}{Catch} has:

\begin{Shaded}
\begin{Highlighting}[]
    \NormalTok{REQUIRE(expression); }\CommentTok{// stop if fail}
    \NormalTok{CHECK(expression);   }\CommentTok{// doesn't stop if fails}
\end{Highlighting}
\end{Shaded}

If an exception is thrown, it's caught, reported and counts as a
failure.

Examples:

\begin{Shaded}
\begin{Highlighting}[]
    \NormalTok{CHECK( str == }\StringTok{"string value"} \NormalTok{);}
    \NormalTok{CHECK( thisReturnsTrue() );}
    \NormalTok{REQUIRE( i == }\DecValTok{42} \NormalTok{);}
\end{Highlighting}
\end{Shaded}

Others:

\begin{Shaded}
\begin{Highlighting}[]
    \NormalTok{REQUIRE_FALSE( expression )}
    \NormalTok{CHECK_FALSE( expression )}
    \NormalTok{REQUIRE_THROWS( expression ) # Must }\KeywordTok{throw} \NormalTok{an exception}
    \NormalTok{CHECK_THROWS( expression ) # Must }\KeywordTok{throw} \NormalTok{an exception, }\KeywordTok{and} \KeywordTok{continue} \NormalTok{testing}
    \NormalTok{REQUIRE_THROWS_AS( expression, exception type )}
    \NormalTok{CHECK_THROWS_AS( expression, exception type )}
    \NormalTok{REQUIRE_NOTHROW( expression )}
    \NormalTok{CHECK_NOTHROW( expression )}
\end{Highlighting}
\end{Shaded}

\subsubsection{Testing for Failure}\label{testing-for-failure}

To re-iterate:

\begin{itemize}
\itemsep1pt\parskip0pt\parsep0pt
\item
  You should test failure cases

  \begin{itemize}
  \itemsep1pt\parskip0pt\parsep0pt
  \item
    Force a failure
  \item
    Check that exception is thrown
  \item
    If exception is thrown, test passes
  \item
    (Some people get confused, expecting test to fail)
  \end{itemize}
\item
  Examples

  \begin{itemize}
  \itemsep1pt\parskip0pt\parsep0pt
  \item
    Saving to invalid file name
  \item
    Negative numbers passed into double arguments
  \item
    Invalid Physical quantities (e.g. -300 Kelvin)
  \end{itemize}
\end{itemize}

\subsubsection{Setup/Tear Down}\label{setuptear-down}

\begin{itemize}
\itemsep1pt\parskip0pt\parsep0pt
\item
  Some tests require objects to exist in memory
\item
  These should be set up

  \begin{itemize}
  \itemsep1pt\parskip0pt\parsep0pt
  \item
    for each test
  \item
    for a group of tests
  \end{itemize}
\item
  Frameworks do differ in this regards
\end{itemize}

\subsubsection{Setup/Tear Down in Catch}\label{setuptear-down-in-catch}

Referring to the
\href{https://github.com/philsquared/Catch/blob/master/docs/tutorial.md}{Catch
Tutorial}:

\begin{Shaded}
\begin{Highlighting}[]
\NormalTok{TEST_CASE( }\StringTok{"vectors can be sized and resized"}\NormalTok{, }\StringTok{"[vector]"} \NormalTok{) \{}

    \NormalTok{std::vector<}\DataTypeTok{int}\NormalTok{> v( }\DecValTok{5} \NormalTok{);}

    \NormalTok{REQUIRE( v.size() == }\DecValTok{5} \NormalTok{);}
    \NormalTok{REQUIRE( v.capacity() >= }\DecValTok{5} \NormalTok{);}

    \NormalTok{SECTION( }\StringTok{"resizing bigger changes size and capacity"} \NormalTok{) \{}
        \NormalTok{v.resize( }\DecValTok{10} \NormalTok{);}

        \NormalTok{REQUIRE( v.size() == }\DecValTok{10} \NormalTok{);}
        \NormalTok{REQUIRE( v.capacity() >= }\DecValTok{10} \NormalTok{);}
    \NormalTok{\}}
    \NormalTok{SECTION( }\StringTok{"resizing smaller changes size but not capacity"} \NormalTok{) \{}
        \NormalTok{v.resize( }\DecValTok{0} \NormalTok{);}

        \NormalTok{REQUIRE( v.size() == }\DecValTok{0} \NormalTok{);}
        \NormalTok{REQUIRE( v.capacity() >= }\DecValTok{5} \NormalTok{);}
    \NormalTok{\}}
    \NormalTok{SECTION( }\StringTok{"reserving bigger changes capacity but not size"} \NormalTok{) \{}
        \NormalTok{v.reserve( }\DecValTok{10} \NormalTok{);}

        \NormalTok{REQUIRE( v.size() == }\DecValTok{5} \NormalTok{);}
        \NormalTok{REQUIRE( v.capacity() >= }\DecValTok{10} \NormalTok{);}
    \NormalTok{\}}
    \NormalTok{SECTION( }\StringTok{"reserving smaller does not change size or capacity"} \NormalTok{) \{}
        \NormalTok{v.reserve( }\DecValTok{0} \NormalTok{);}

        \NormalTok{REQUIRE( v.size() == }\DecValTok{5} \NormalTok{);}
        \NormalTok{REQUIRE( v.capacity() >= }\DecValTok{5} \NormalTok{);}
    \NormalTok{\}}
\NormalTok{\}}
\end{Highlighting}
\end{Shaded}

So, Setup/Tear down is done before/after each section.

\subsection{Unit Testing Tips}\label{unit-testing-tips}

\subsubsection{C++ design}\label{c-design}

\begin{itemize}
\itemsep1pt\parskip0pt\parsep0pt
\item
  Stuff from above applies to Classes / Functions
\item
  Think about arguments:

  \begin{itemize}
  \itemsep1pt\parskip0pt\parsep0pt
  \item
    Code should be hard to use incorrectly.
  \item
    Use \texttt{const}, \texttt{unsigned} etc.
  \item
    Testing forces you to sort these out.
  \end{itemize}
\end{itemize}

\subsubsection{Test Driven Development
(TDD)}\label{test-driven-development-tdd}

\begin{itemize}
\itemsep1pt\parskip0pt\parsep0pt
\item
  Methodology

  \begin{enumerate}
  \def\labelenumi{\arabic{enumi}.}
  \itemsep1pt\parskip0pt\parsep0pt
  \item
    Write a test
  \item
    Run test, should fail
  \item
    Implement/Debug functionality
  \item
    Run test

    \begin{enumerate}
    \def\labelenumii{\arabic{enumii}.}
    \itemsep1pt\parskip0pt\parsep0pt
    \item
      if succeed goto 5
    \item
      else goto 3
    \end{enumerate}
  \item
    Refactor to tidy up
  \end{enumerate}
\end{itemize}

\subsubsection{TDD in practice}\label{tdd-in-practice}

\begin{itemize}
\itemsep1pt\parskip0pt\parsep0pt
\item
  Aim to get good coverage
\item
  Some people quote 70\% or more
\item
  What are the downsides?
\item
  Don't write `brittle' tests
\end{itemize}

\subsubsection{Behaviour Driven Development
(BDD)}\label{behaviour-driven-development-bdd}

\begin{itemize}
\itemsep1pt\parskip0pt\parsep0pt
\item
  Behaviour Driven Development (BDD)

  \begin{itemize}
  \itemsep1pt\parskip0pt\parsep0pt
  \item
    Refers to a
    \href{https://en.wikipedia.org/wiki/Behavior-driven_development}{whole
    area} of software engineering
  \item
    With associated tools and practices
  \item
    Think about end-user perspective
  \item
    Think about the desired behaviour not the implementation
  \item
    See
    \href{https://codeutopia.net/blog/2015/03/01/unit-testing-tdd-and-bdd/}{Jani
    Hartikainen} article.
  \end{itemize}
\end{itemize}

\subsubsection{TDD Vs BDD}\label{tdd-vs-bdd}

\begin{itemize}
\itemsep1pt\parskip0pt\parsep0pt
\item
  TDD

  \begin{itemize}
  \itemsep1pt\parskip0pt\parsep0pt
  \item
    Test/Design based on methods available
  \item
    Often ties in implementation details
  \end{itemize}
\item
  BDD

  \begin{itemize}
  \itemsep1pt\parskip0pt\parsep0pt
  \item
    Test/Design based on behaviour\\
  \item
    Code to interfaces (later in course)
  \end{itemize}
\item
  Subtly different
\item
  Aim for BDD
\end{itemize}

\subsubsection{Anti-Pattern 1:
Setters/Getters}\label{anti-pattern-1-settersgetters}

Testing every Setter/Getter.

Consider:

\begin{Shaded}
\begin{Highlighting}[]
   \KeywordTok{class} \NormalTok{Atom \{}

     \KeywordTok{public}\NormalTok{:}
       \DataTypeTok{void} \NormalTok{SetAtomicNumber(}\DataTypeTok{const} \DataTypeTok{int}\NormalTok{& number) \{ m_AtomicNumber = number; \}}
       \DataTypeTok{int} \NormalTok{GetAtomicNumber() }\DataTypeTok{const} \NormalTok{\{ }\KeywordTok{return} \NormalTok{m_AtomicNumber; \}}
       \DataTypeTok{void} \NormalTok{SetName(}\DataTypeTok{const} \NormalTok{std::string& name) \{ m_Name = name; \}}
       \NormalTok{std::string GetName() }\DataTypeTok{const} \NormalTok{\{ }\KeywordTok{return} \NormalTok{m_Name; \}}
     \KeywordTok{private}\NormalTok{:}
       \DataTypeTok{int} \NormalTok{m_AtomicNumber;}
       \NormalTok{std::string m_Name;}
   \NormalTok{\};}
\end{Highlighting}
\end{Shaded}

and tests like:

\begin{Shaded}
\begin{Highlighting}[]
    \NormalTok{TEST_CASE( }\StringTok{"Testing Setters/Getters"}\NormalTok{, }\StringTok{"[Atom]"} \NormalTok{) \{}

        \NormalTok{Atom a;}

        \NormalTok{a.SetAtomicNumber(}\DecValTok{1}\NormalTok{);}
        \NormalTok{REQUIRE( a.GetAtomicNumber() == }\DecValTok{1}\NormalTok{);}
        \NormalTok{a.SetName(}\StringTok{"Hydrogen"}\NormalTok{);}
        \NormalTok{REQUIRE( a.GetName() == }\StringTok{"Hydrogen"}\NormalTok{);}
\end{Highlighting}
\end{Shaded}

\begin{itemize}
\itemsep1pt\parskip0pt\parsep0pt
\item
  It feels tedious
\item
  But you want good coverage
\item
  This often puts people off testing
\item
  It also produces ``brittle'', where 1 change breaks many things
\end{itemize}

\subsubsection{Anti-Pattern 1:
Suggestion.}\label{anti-pattern-1-suggestion.}

\begin{itemize}
\itemsep1pt\parskip0pt\parsep0pt
\item
  Focus on behaviour.

  \begin{itemize}
  \itemsep1pt\parskip0pt\parsep0pt
  \item
    What would end-user expect to be doing?
  \item
    How would end-user be using this class?
  \item
    Write tests that follow the use-case
  \item
    Gives a more logical grouping
  \item
    One test can cover \textgreater{} 1 function
  \item
    i.e.~move away from slavishly testing each function
  \end{itemize}
\item
  Minimise interface.

  \begin{itemize}
  \itemsep1pt\parskip0pt\parsep0pt
  \item
    Provide the bare number of methods
  \item
    Don't provide setters if you don't want them
  \item
    Don't provide getters unless the user needs something
  \item
    Less to test. Use documentation to describe why.
  \end{itemize}
\end{itemize}

\subsubsection{Anti-Pattern 2: Constructing Dependent
Classes}\label{anti-pattern-2-constructing-dependent-classes}

\begin{itemize}
\itemsep1pt\parskip0pt\parsep0pt
\item
  Sometimes, by necessity we test groups of classes
\item
  Or one class genuinely Has-A contained class
\item
  But the contained class is expensive, or could be changed in future
\end{itemize}

\subsubsection{Anti-Pattern 2:
Suggestion}\label{anti-pattern-2-suggestion}

\begin{itemize}
\itemsep1pt\parskip0pt\parsep0pt
\item
  Read up on
  \href{https://martinfowler.com/articles/injection.html}{Dependency
  Injection}
\item
  Enables you to create and inject dummy test classes
\item
  So, testing again used to break down design, and increase flexibility
\end{itemize}

\subsubsection{Summary BDD Vs TDD}\label{summary-bdd-vs-tdd}

Aim to write:

\begin{itemize}
\itemsep1pt\parskip0pt\parsep0pt
\item
  Most concise description of requirements as unit tests
\item
  Smallest amount of code to pass tests
\item
  \ldots{} i.e.~based on behaviour
\end{itemize}

\subsection{Any Questions?}\label{any-questions}

\subsubsection{Homework - Overview}\label{homework---overview}

\begin{itemize}
\itemsep1pt\parskip0pt\parsep0pt
\item
  Example git repo, CMake, Catch template project:

  \begin{itemize}
  \itemsep1pt\parskip0pt\parsep0pt
  \item
    \href{https://github.com/MattClarkson/CMakeCatch2}{CMakeCatch2} -
    Simple
  \item
    \href{https://github.com/MattClarkson/CMakeCatchTemplate}{CMakeCatchTemplate}
    - Complex
  \end{itemize}
\item
  You should

  \begin{itemize}
  \itemsep1pt\parskip0pt\parsep0pt
  \item
    Clone, Build.
  \item
    Add unit test in Testing
  \item
    Run via ctest
  \item
    Find log file
  \end{itemize}
\end{itemize}

\subsubsection{Homework - 7}\label{homework---7}

\begin{itemize}
\itemsep1pt\parskip0pt\parsep0pt
\item
  Imagine a simple function, e.g.~to add two numbers.
\item
  Play with unit tests until you understand the difference between:
\end{itemize}

\begin{Shaded}
\begin{Highlighting}[]
\DataTypeTok{int} \NormalTok{AddTwoNumbers(}\DataTypeTok{int} \NormalTok{a, }\DataTypeTok{int} \NormalTok{b);}
\DataTypeTok{int} \NormalTok{AddTwoNumbers(}\DataTypeTok{const} \DataTypeTok{int}\NormalTok{& a, }\DataTypeTok{const} \DataTypeTok{int}\NormalTok{&b);}
\DataTypeTok{void} \NormalTok{AddTwoNumbers(}\DataTypeTok{int}\NormalTok{* a, }\DataTypeTok{int}\NormalTok{*b, }\DataTypeTok{int}\NormalTok{* output);}
\DataTypeTok{void} \NormalTok{AddTwoNumbers(}\DataTypeTok{const} \DataTypeTok{int}\NormalTok{* }\DataTypeTok{const} \NormalTok{a, }\DataTypeTok{const} \DataTypeTok{int}\NormalTok{* }\DataTypeTok{const} \NormalTok{b);}
\end{Highlighting}
\end{Shaded}

Now imagine, instead of integers, the variables all contained a large
Image. Which type of function declaration would you use?

\subsubsection{Homework - 8}\label{homework---8}

\begin{itemize}
\itemsep1pt\parskip0pt\parsep0pt
\item
  Write a Fraction class
\item
  Write a print function to print nicely formatted fractions
\item
  Does the print function live inside or outside of the class?
\item
  Write a method \texttt{simplify()} which will simplify the fraction.
\item
  Unit test until you have at least got the hang of unit testing
\item
  Review your function arguments and return types
\end{itemize}

\section{Lecture 2: Modern C++ (1)}\label{lecture-2-modern-c-1}

\subsection{Lecture 2: Overview}\label{lecture-2-overview}

\subsubsection{The Story So Far}\label{the-story-so-far}

\begin{itemize}
\itemsep1pt\parskip0pt\parsep0pt
\item
  Git
\item
  CMake
\item
  Catch2
\end{itemize}

\subsubsection{Todays Lesson}\label{todays-lesson}

\begin{itemize}
\itemsep1pt\parskip0pt\parsep0pt
\item
  Recap of C++ features
\item
  OO concepts:

  \begin{itemize}
  \itemsep1pt\parskip0pt\parsep0pt
  \item
    Encapsulation and data abstraction
  \item
    Inheritance
  \item
    Polymorphism
  \end{itemize}
\end{itemize}

\subsection{Recap of C++ features}\label{recap-of-c-features}

\subsubsection{C++ is evolving}\label{c-is-evolving}

\begin{itemize}
\itemsep1pt\parskip0pt\parsep0pt
\item
  C++98
\item
  Introduced templates
\item
  STL containers and algorithms
\item
  Strings and IO/streams\\
\item
  C++11
\item
  Many new features introduced, feels like a different programming
  language
\item
  Move semantics
\item
  \texttt{auto}
\item
  Lambda functions
\item
  \texttt{constexpr}
\item
  Smart pointers
\item
  \texttt{std::array}
\item
  Support for multithreading
\item
  Regular expressions
\item
  C++14
\item
  \texttt{auto} works in more places
\item
  Generalised lambda functions
\item
  \texttt{std::make\_unique}
\item
  Reader/writer locks
\item
  C++17
\item
  fold expressions
\item
  \texttt{std::any} and \texttt{std::variant}
\item
  The Filesystem library
\item
  and more\ldots{}
\end{itemize}

C++ constantly evolving, you don't just learn once and then stop you
need keep up with language developments. For this course we will be
using up until C++14.

\subsubsection{Some excercises using}\label{some-excercises-using}

\begin{itemize}
\itemsep1pt\parskip0pt\parsep0pt
\item
  Now some excercises manipulating \texttt{vectors}, using range based
  for loops and some some algorithms from
  \texttt{\textless{}algorithm\textgreater{}}.
\end{itemize}

\subsubsection{Homework - 9}\label{homework---9}

\begin{itemize}
\itemsep1pt\parskip0pt\parsep0pt
\item
  Use \url{https://github.com/MattClarkson/CMakeLibraryAndApp.git} and
  create a new library to perform various operations on a std::vector
\item
  Using a range base for loop Write a function that prints all the
  elements on the vector to screen and call this from \texttt{myApp}
\item
  Write a function using a range based for loop that counts the number
  of elements equal to a value 5
\item
  Now repeat but instead use the \texttt{std::count} algorithm from the
  \texttt{\textless{}algorithm\textgreater{}} library
\end{itemize}

\subsubsection{Homework - 10}\label{homework---10}

\begin{itemize}
\itemsep1pt\parskip0pt\parsep0pt
\item
  Again using
  \url{https://github.com/MattClarkson/CMakeLibraryAndApp.git}
\item
  Write a function
  \texttt{add\_elements(vector\textless{}int\textgreater{} v, int val, int ntimes)}
  that takes a \texttt{vector\textless{}int\textgreater{}} and appends
  \texttt{ntimes} new elements with value \texttt{val}\\
\item
  Print contents of the input vector to screen before and after calling
  the function as well from within the \texttt{add\_elements} function,
  what is the problem?
\item
  Try passing by reference
  \texttt{add\_elements(vector\textless{}int\textgreater{} \&v, ...}
  instead, does it work now?
\item
  What are the advantages/disadvantages to passing by reference?
\item
  Try passing as a \texttt{const} reference
  \texttt{add\_elements(const vector\textless{}int\textgreater{} \&v, ...},
  what happens now and why would you use this?
\end{itemize}

\subsection{Object Oriented Review}\label{object-oriented-review}

\subsubsection{C-style Programming}\label{c-style-programming}

\begin{itemize}
\itemsep1pt\parskip0pt\parsep0pt
\item
  Procedural programming
\item
  Pass data to functions
\end{itemize}

\subsubsection{Function-style Example}\label{function-style-example}

\begin{Shaded}
\begin{Highlighting}[]

\DataTypeTok{double} \NormalTok{compute_similarity(}\DataTypeTok{double} \NormalTok{*imag1, }\DataTypeTok{double}\NormalTok{* image2, }\DataTypeTok{double} \NormalTok{*params)}
\NormalTok{\{}
  \CommentTok{// stuff}
\NormalTok{\}}

\DataTypeTok{double} \NormalTok{calculate_derivative(}\DataTypeTok{double}\NormalTok{* params)}
\NormalTok{\{}
  \CommentTok{// stuff}
\NormalTok{\}}

\DataTypeTok{double} \NormalTok{convert_to_decimal(}\DataTypeTok{double} \NormalTok{numerator, }\DataTypeTok{double} \NormalTok{denominator)}
\NormalTok{\{}
  \CommentTok{// stuff}
\NormalTok{\}}
\end{Highlighting}
\end{Shaded}

\subsubsection{Disadvantages}\label{disadvantages}

\begin{itemize}
\itemsep1pt\parskip0pt\parsep0pt
\item
  Can get out of hand as program size increases
\item
  Can't easily describe relationships between bits of data
\item
  Relies on method documentation, function and variable names
\item
  Can't easily control/(enforce control of) access to data
\end{itemize}

\subsubsection{C Struct}\label{c-struct}

\begin{itemize}
\itemsep1pt\parskip0pt\parsep0pt
\item
  So, in C, the struct was invented
\item
  Basically a class without methods
\item
  This at least provides a logical grouping
\end{itemize}

\subsubsection{Struct Example}\label{struct-example}

\begin{Shaded}
\begin{Highlighting}[]

\KeywordTok{struct} \NormalTok{Fraction \{}
  \DataTypeTok{int} \NormalTok{numerator;}
  \DataTypeTok{int} \NormalTok{denominator;}
\NormalTok{\};}

\DataTypeTok{double} \NormalTok{convertToDecimal(}\DataTypeTok{const} \NormalTok{Fraction& f)}
\NormalTok{\{}
  \KeywordTok{return} \NormalTok{f.numerator/}\KeywordTok{static_cast}\NormalTok{<}\DataTypeTok{double}\NormalTok{>(f.denominator); }
\NormalTok{\}}
\end{Highlighting}
\end{Shaded}

\subsubsection{C++ Class}\label{c-class}

\begin{itemize}
\itemsep1pt\parskip0pt\parsep0pt
\item
  C++ provides the class to enhance the language with user defined types
\item
  Once defined, use types as if native to the language
\end{itemize}

\subsubsection{Abstraction}\label{abstraction}

\begin{itemize}
\itemsep1pt\parskip0pt\parsep0pt
\item
  C++ class mechanism enables you to define a type

  \begin{itemize}
  \itemsep1pt\parskip0pt\parsep0pt
  \item
    independent of its data
  \item
    independent of its implementation
  \item
    class defines concept or blueprint
  \item
    instantiation creates object
  \end{itemize}
\end{itemize}

\subsubsection{Abstraction - Grady
Booch}\label{abstraction---grady-booch}

``An abstraction denotes the essential characteristics of an object that
distinguish it from all other kinds of objects and thus provide crisply
defined conceptual boundaries, relative to the perspective of the
viewer.''

\subsubsection{Class Example}\label{class-example}

\begin{Shaded}
\begin{Highlighting}[]
\KeywordTok{class} \NormalTok{Atom \{}
\KeywordTok{public}\NormalTok{:}
  \NormalTok{Atom()\{\};}
  \NormalTok{~Atom()\{\};}
  \DataTypeTok{int} \NormalTok{GetAtomicNumber();}
  \DataTypeTok{double} \NormalTok{GetAtomicWeight();}
\NormalTok{\};}

\DataTypeTok{int} \NormalTok{main(}\DataTypeTok{int} \NormalTok{argc, }\DataTypeTok{char}\NormalTok{** argv)}
\NormalTok{\{}
  \NormalTok{Atom a;}
\NormalTok{\}}
\end{Highlighting}
\end{Shaded}

\subsubsection{Encapsulation}\label{encapsulation}

\begin{itemize}
\itemsep1pt\parskip0pt\parsep0pt
\item
  Encapsulation is:

  \begin{itemize}
  \itemsep1pt\parskip0pt\parsep0pt
  \item
    Bundling together methods and data
  \item
    Restricting access, defining public interface
  \end{itemize}
\item
  Describes how you correctly use something
\end{itemize}

\subsubsection{Public/Private/Protected}\label{publicprivateprotected}

\begin{itemize}
\itemsep1pt\parskip0pt\parsep0pt
\item
  For class methods/variables:

  \begin{itemize}
  \itemsep1pt\parskip0pt\parsep0pt
  \item
    \texttt{private}: only available in this class
  \item
    \texttt{protected}: available in this class and derived classes
  \item
    \texttt{public}: available to anyone with access to the object
  \end{itemize}
\item
  (public, protected, private inheritance comes later)
\end{itemize}

\subsubsection{Class Example}\label{class-example-1}

\begin{Shaded}
\begin{Highlighting}[]
\CommentTok{// Defining a User Defined Type.}
\KeywordTok{class} \NormalTok{Fraction \{}

\KeywordTok{public}\NormalTok{: }\CommentTok{// access control}
  \CommentTok{// How to create}
  \NormalTok{Fraction();}
  \NormalTok{Fraction(}\DataTypeTok{const} \DataTypeTok{int} \NormalTok{&num, }\DataTypeTok{const} \DataTypeTok{int} \NormalTok{&denom);}

  \CommentTok{// How to destroy}
  \NormalTok{~Fraction();}

  \CommentTok{// How to access}
  \DataTypeTok{int} \NormalTok{numerator() }\DataTypeTok{const}\NormalTok{;}
  \DataTypeTok{int} \NormalTok{denominator() }\DataTypeTok{const}\NormalTok{;}

  \CommentTok{// What you can do}
  \DataTypeTok{const} \NormalTok{Fraction }\KeywordTok{operator}\NormalTok{+(}\DataTypeTok{const} \NormalTok{Fraction &another);}

\KeywordTok{private}\NormalTok{: }\CommentTok{// access control}
  \CommentTok{// The data}
  \DataTypeTok{int} \NormalTok{m_Numerator;}
  \DataTypeTok{int} \NormalTok{m_Denominator;}
\NormalTok{\};}
\end{Highlighting}
\end{Shaded}

\subsubsection{Inheritance}\label{inheritance}

\begin{itemize}
\itemsep1pt\parskip0pt\parsep0pt
\item
  Used for:

  \begin{itemize}
  \itemsep1pt\parskip0pt\parsep0pt
  \item
    Defining new types based on a common type
  \end{itemize}
\item
  Careful:

  \begin{itemize}
  \itemsep1pt\parskip0pt\parsep0pt
  \item
    Beware - ``Reduce code duplication, less maintenance''
  \item
    Types in a hierarchy MUST be related
  \item
    Don't over-use inheritance
  \item
    We will cover other ways of object re-use
  \end{itemize}
\end{itemize}

\subsubsection{Class Example}\label{class-example-2}

\begin{Shaded}
\begin{Highlighting}[]
\KeywordTok{class} \NormalTok{Shape \{}
\KeywordTok{public}\NormalTok{:}
  \NormalTok{Shape();}
  \DataTypeTok{void} \NormalTok{setVisible(}\DataTypeTok{const} \DataTypeTok{bool} \NormalTok{&isVisible) \{ m_IsVisible = isVisible; \}}
  \KeywordTok{virtual} \DataTypeTok{void} \NormalTok{rotate(}\DataTypeTok{const} \DataTypeTok{double} \NormalTok{&degrees) = }\DecValTok{0}\NormalTok{;}
  \KeywordTok{virtual} \DataTypeTok{void} \NormalTok{scale(}\DataTypeTok{const} \DataTypeTok{double} \NormalTok{&factor) = }\DecValTok{0}\NormalTok{;}
  \CommentTok{// + other methods}
\KeywordTok{private}\NormalTok{:}
  \DataTypeTok{bool} \NormalTok{m_IsVisible;}
  \DataTypeTok{unsigned} \DataTypeTok{char} \NormalTok{m_Colour[}\DecValTok{3}\NormalTok{]; }\CommentTok{// RGB}
  \DataTypeTok{double} \NormalTok{m_CentreOfMass[}\DecValTok{2}\NormalTok{];}
\NormalTok{\};}

\KeywordTok{class} \NormalTok{Rectangle : }\KeywordTok{public} \NormalTok{Shape \{}
\KeywordTok{public}\NormalTok{:}
  \NormalTok{Rectangle();}
  \KeywordTok{virtual} \DataTypeTok{void} \NormalTok{rotate(}\DataTypeTok{const} \DataTypeTok{double} \NormalTok{&degrees);}
  \KeywordTok{virtual} \DataTypeTok{void} \NormalTok{scale(}\DataTypeTok{const} \DataTypeTok{double} \NormalTok{&factor);}
  \CommentTok{// + other methods}
\KeywordTok{private}\NormalTok{:}
  \DataTypeTok{double} \NormalTok{m_Corner1[}\DecValTok{2}\NormalTok{];}
  \DataTypeTok{double} \NormalTok{m_Corner2[}\DecValTok{2}\NormalTok{];}
\NormalTok{\};}

\KeywordTok{class} \NormalTok{Circle : }\KeywordTok{public} \NormalTok{Shape \{}
\KeywordTok{public}\NormalTok{:}
  \NormalTok{Circle();}
  \KeywordTok{virtual} \DataTypeTok{void} \NormalTok{rotate(}\DataTypeTok{const} \DataTypeTok{double} \NormalTok{&degrees);}
  \KeywordTok{virtual} \DataTypeTok{void} \NormalTok{scale(}\DataTypeTok{const} \DataTypeTok{double} \NormalTok{&factor);}
  \CommentTok{// + other methods}
\KeywordTok{private}\NormalTok{:}
  \DataTypeTok{float} \NormalTok{radius;}
\NormalTok{\};}
\end{Highlighting}
\end{Shaded}

\subsubsection{Polymorphism}\label{polymorphism}

\begin{itemize}
\itemsep1pt\parskip0pt\parsep0pt
\item
  Several types:

  \begin{itemize}
  \itemsep1pt\parskip0pt\parsep0pt
  \item
    (normally) ``subtype'': via inheritance
  \item
    ``parametric'': via templates
  \item
    ``ad hoc'': via function overloading
  \end{itemize}
\item
  Common interface to entities of different types
\item
  Same method, different behaviour
\end{itemize}

\subsubsection{Class Example}\label{class-example-3}

\begin{Shaded}
\begin{Highlighting}[]
\OtherTok{#include "shape.h"}
\DataTypeTok{int} \NormalTok{main(}\DataTypeTok{int} \NormalTok{argc, }\DataTypeTok{char}\NormalTok{** argv)}
\NormalTok{\{}
  \NormalTok{Circle c1;}
  \NormalTok{Rectangle r1;}
  \NormalTok{Shape *s1 = &c1;}
  \NormalTok{Shape *s2 = &r1;}

  \CommentTok{// Calls method in Shape (as not virtual)}
  \DataTypeTok{bool} \NormalTok{isVisible = }\KeywordTok{true}\NormalTok{;}
  \NormalTok{s1->setVisible(isVisible);}
  \NormalTok{s2->setVisible(isVisible);}

  \CommentTok{// Calls method in derived (as declared virtual)}
  \NormalTok{s1->rotate(}\DecValTok{10}\NormalTok{);}
  \NormalTok{s2->rotate(}\DecValTok{10}\NormalTok{);}
\NormalTok{\}}
\end{Highlighting}
\end{Shaded}

\subsubsection{Homework - 11}\label{homework---11}

\begin{itemize}
\itemsep1pt\parskip0pt\parsep0pt
\item
  Use \url{https://github.com/MattClarkson/CMakeCatch2.git} and create a
  simple shape class for a square with methods to calculate the area
\item
  Create an object of the class from within an app and print the result
  of the area calculation to screen
\item
  Try to write some unit tests to check the class behaves as expected
\end{itemize}

\subsubsection{Homework - 12}\label{homework---12}

\begin{itemize}
\itemsep1pt\parskip0pt\parsep0pt
\item
  Again using \texttt{CMakeCatch2} as a basis use inheritance and
  polymorphism to create a shape base class and a set of derived classes
  for a square, rectangle
\item
  Draw a class inheritance diagram before starting
\item
  Confirm that the get area functions behave differently for the
  different dervied classes
\end{itemize}

\section{Lecture 3: Modern C++ (2)}\label{lecture-3-modern-c-2}

\subsection{Lecture 3: Overview}\label{lecture-3-overview}

\subsubsection{The Story So Far}\label{the-story-so-far-1}

\begin{itemize}
\itemsep1pt\parskip0pt\parsep0pt
\item
  Git, CMake, Catch2
\item
  Recap of Modern C++ features
\item
  OO concepts
\end{itemize}

\subsubsection{Todays Lesson}\label{todays-lesson-1}

\begin{itemize}
\itemsep1pt\parskip0pt\parsep0pt
\item
  C++ Standard Library
\item
  Smart pointers and move semantics
\item
  Lambda expressions
\item
  Exceptions
\end{itemize}

\subsection{C++ Standard Library}\label{c-standard-library}

\subsubsection{What is it}\label{what-is-it}

\begin{itemize}
\item
  A collection of:
\item
  Containers (e.g. \texttt{std::vector}, \texttt{std::array},
  \texttt{std::map}, \ldots{})
\item
  Methods to manipulate these containers (e.g. \texttt{std::sort},
  \texttt{std::find\_if})
\item
  Strings (\texttt{std::string}) and streams (e.g. \texttt{std::cout},
  \texttt{std::endl}, \ldots{})
\item
  Function objects (e.g.~arithmetic operations, comparisons and logical
  operations, \ldots{})
\item
  Part of the ISO Standard itself so it has to be implemented to be
  called C++
\item
  Can include in your code by simply adding the relevant
  \texttt{\#include \textless{}headername\textgreater{}} and then using
  the \texttt{std::} namespace
\item
  A list of these headers can be found at
  \url{https://en.cppreference.com/w/cpp/header}
\end{itemize}

\subsubsection{Homework - 13}\label{homework---13}

\begin{itemize}
\itemsep1pt\parskip0pt\parsep0pt
\item
  Write short code snippets that use:
\item
  Containers \texttt{\textless{}array\textgreater{}},
  \texttt{\textless{}vector\textgreater{}} and
  \texttt{\textless{}map\textgreater{}}\\
\item
  The \texttt{\textless{}random\textgreater{}} number generator library
\item
  The \texttt{\textless{}iostream\textgreater{}} to read in a string
  from the terminal and output it to screen
\end{itemize}

\subsection{Smart Pointers}\label{smart-pointers}

\subsubsection{Use of Raw Pointers}\label{use-of-raw-pointers}

\begin{itemize}
\itemsep1pt\parskip0pt\parsep0pt
\item
  Given a pointer passed to a function
\end{itemize}

\begin{verbatim}
   void DoSomethingClever(int *a) 
   {
     // write some code
   }
\end{verbatim}

\begin{itemize}
\itemsep1pt\parskip0pt\parsep0pt
\item
  How do we use the pointer?
\item
  What problems are there?
\end{itemize}

\subsubsection{Problems with Raw
Pointers}\label{problems-with-raw-pointers}

\begin{itemize}
\itemsep1pt\parskip0pt\parsep0pt
\item
  From
  \href{https://www.amazon.co.uk/Effective-Modern-Specific-Ways-Improve/dp/1491903996/ref=sr_1_1?ie=UTF8\&qid=1484571499\&sr=8-1\&keywords=Effective+Modern+C\%2B\%2B}{``Effective
  Modern C++'', Meyers, p117}.

  \begin{itemize}
  \itemsep1pt\parskip0pt\parsep0pt
  \item
    If you are done, do you destroy it?
  \item
    How to destroy it? Call \texttt{delete} or some method first:
    \texttt{a-\textgreater{}Shutdown();}
  \item
    Single object or array?
  \item
    \texttt{delete} or \texttt{delete{[}{]}}?
  \item
    How to ensure the whole system only deletes it once?
  \item
    Is it dangling, if I don't delete it?
  \end{itemize}
\end{itemize}

\subsubsection{Use Smart Pointers}\label{use-smart-pointers}

\begin{itemize}
\itemsep1pt\parskip0pt\parsep0pt
\item
  \texttt{new/delete} on raw pointers not good enough
\item
  So, use Smart Pointers

  \begin{itemize}
  \itemsep1pt\parskip0pt\parsep0pt
  \item
    automatically delete pointed to object
  \item
    explicit control over sharing
  \item
    i.e.~smarter
  \end{itemize}
\item
  Smart Pointers model the ``ownership''
\end{itemize}

\subsubsection{Further Reading}\label{further-reading}

\begin{itemize}
\itemsep1pt\parskip0pt\parsep0pt
\item
  Notes here are based on these:

  \begin{itemize}
  \itemsep1pt\parskip0pt\parsep0pt
  \item
    \href{http://www.umich.edu/~eecs381/handouts/C++11_smart_ptrs.pdf}{David
    Kieras online paper}
  \item
    \href{https://www.amazon.co.uk/Effective-Modern-Specific-Ways-Improve/dp/1491903996/ref=sr_1_1?ie=UTF8\&qid=1484571499\&sr=8-1\&keywords=Effective+Modern+C\%2B\%2B}{``Effective
    Modern C++'', Meyers, ch4}
  \end{itemize}
\end{itemize}

\subsubsection{Standard Library Smart
Pointers}\label{standard-library-smart-pointers}

\begin{itemize}
\itemsep1pt\parskip0pt\parsep0pt
\item
  Here we teach Standard Library

  \begin{itemize}
  \itemsep1pt\parskip0pt\parsep0pt
  \item
    \href{http://en.cppreference.com/w/cpp/memory/unique_ptr}{std::unique\_ptr}
    - models \emph{has-a} but also unique ownership
  \item
    \href{http://en.cppreference.com/w/cpp/memory/shared_ptr}{std::shared\_ptr}
    - models \emph{has-a} but shared ownership
  \item
    \href{http://en.cppreference.com/w/cpp/memory/weak_ptr}{std::weak\_ptr}
    - temporary reference, breaks circular references
  \end{itemize}
\end{itemize}

\subsubsection{Stack Allocated - No
Leak.}\label{stack-allocated---no-leak.}

\begin{itemize}
\itemsep1pt\parskip0pt\parsep0pt
\item
  To recap:
\end{itemize}

\begin{Shaded}
\begin{Highlighting}[]
\OtherTok{#include "Fraction.h"}
\DataTypeTok{int} \NormalTok{main() \{}
  \NormalTok{Fraction f(}\DecValTok{1}\NormalTok{,}\DecValTok{4}\NormalTok{);}
\NormalTok{\}}

\end{Highlighting}
\end{Shaded}

\begin{itemize}
\itemsep1pt\parskip0pt\parsep0pt
\item
  Gives:
\end{itemize}

\begin{verbatim}
I'm being deleted
\end{verbatim}

\begin{itemize}
\itemsep1pt\parskip0pt\parsep0pt
\item
  So stack allocated objects are deleted, when stack unwinds.
\end{itemize}

\subsubsection{Heap Allocated - Leak.}\label{heap-allocated---leak.}

\begin{itemize}
\itemsep1pt\parskip0pt\parsep0pt
\item
  To recap:
\end{itemize}

\begin{Shaded}
\begin{Highlighting}[]
\OtherTok{#include "Fraction.h"}
\DataTypeTok{int} \NormalTok{main() \{}
  \NormalTok{Fraction *f = }\KeywordTok{new} \NormalTok{Fraction(}\DecValTok{1}\NormalTok{,}\DecValTok{4}\NormalTok{);}
\NormalTok{\}}

\end{Highlighting}
\end{Shaded}

\begin{itemize}
\itemsep1pt\parskip0pt\parsep0pt
\item
  Gives:
\end{itemize}

\begin{verbatim}
\end{verbatim}

\begin{itemize}
\itemsep1pt\parskip0pt\parsep0pt
\item
  So heap allocated objects are not deleted.
\item
  Its the pointer (stack allocated) that's deleted.
\end{itemize}

\subsubsection{Unique Ptr - Unique
Ownership}\label{unique-ptr---unique-ownership}

\begin{itemize}
\itemsep1pt\parskip0pt\parsep0pt
\item
  So:
\end{itemize}

\begin{Shaded}
\begin{Highlighting}[]
\OtherTok{#include "Fraction.h"}
\OtherTok{#include <memory>}
\DataTypeTok{int} \NormalTok{main() \{}
  \NormalTok{std::unique_ptr<Fraction> f(}\KeywordTok{new} \NormalTok{Fraction(}\DecValTok{1}\NormalTok{,}\DecValTok{4}\NormalTok{));}
\NormalTok{\}}

\end{Highlighting}
\end{Shaded}

\begin{itemize}
\itemsep1pt\parskip0pt\parsep0pt
\item
  Gives:
\end{itemize}

\begin{verbatim}
I'm being deleted
\end{verbatim}

\begin{itemize}
\itemsep1pt\parskip0pt\parsep0pt
\item
  And object is deleted.
\item
  Is that it?
\end{itemize}

\subsubsection{Unique Ptr - Move?}\label{unique-ptr---move}

\begin{itemize}
\itemsep1pt\parskip0pt\parsep0pt
\item
  Does move work?
\end{itemize}

\begin{Shaded}
\begin{Highlighting}[]
\OtherTok{#include "Fraction.h"}
\OtherTok{#include <memory>}
\OtherTok{#include <iostream>}

\DataTypeTok{int} \NormalTok{main() \{}
  \NormalTok{std::unique_ptr<Fraction> f(}\KeywordTok{new} \NormalTok{Fraction(}\DecValTok{1}\NormalTok{,}\DecValTok{4}\NormalTok{));}
  \CommentTok{// std::unique_ptr<Fraction> f2(f); // compile error}

  \NormalTok{std::cerr << }\StringTok{"f="} \NormalTok{<< f.get() << std::endl;}

  \NormalTok{std::unique_ptr<Fraction> f2;}
  \CommentTok{// f2 = f; // compile error}
  \CommentTok{// f2.reset(f.get()); // bad idea}

  \NormalTok{f2.reset(f.release());}
  \NormalTok{std::cout << }\StringTok{"f="} \NormalTok{<< f.get() << }\StringTok{", f2="} \NormalTok{<< f2.get() << std::endl;}

  \NormalTok{f = std::move(f2);}
  \NormalTok{std::cout << }\StringTok{"f="} \NormalTok{<< f.get() << }\StringTok{", f2="} \NormalTok{<< f2.get() << std::endl;}
\NormalTok{\}}

\end{Highlighting}
\end{Shaded}

\begin{itemize}
\itemsep1pt\parskip0pt\parsep0pt
\item
  Gives:
\end{itemize}

\begin{verbatim}
f=0, f2=0x1cb8010
f=0x1cb8010, f2=0
I'm being deleted
\end{verbatim}

\begin{itemize}
\itemsep1pt\parskip0pt\parsep0pt
\item
  We see that API makes difficult to use incorrectly.
\end{itemize}

\subsubsection{Unique Ptr - Usage 1}\label{unique-ptr---usage-1}

\begin{itemize}
\itemsep1pt\parskip0pt\parsep0pt
\item
  Forces you to think about ownership

  \begin{itemize}
  \itemsep1pt\parskip0pt\parsep0pt
  \item
    No copy constructor
  \item
    No assignment
  \end{itemize}
\item
  Consequently

  \begin{itemize}
  \itemsep1pt\parskip0pt\parsep0pt
  \item
    Can't pass pointer by value
  \item
    Use move semantics for placing in containers
  \end{itemize}
\end{itemize}

\subsubsection{Unique Ptr - Usage 2}\label{unique-ptr---usage-2}

\begin{itemize}
\itemsep1pt\parskip0pt\parsep0pt
\item
  Put raw pointer STRAIGHT into unique\_ptr
\item
  see \texttt{std::make\_unique} in C++14.
\end{itemize}

\begin{Shaded}
\begin{Highlighting}[]
\OtherTok{#include "Fraction.h"}
\OtherTok{#include <memory>}
\DataTypeTok{int} \NormalTok{main() \{}
  \NormalTok{std::unique_ptr<Fraction> f(}\KeywordTok{new} \NormalTok{Fraction(}\DecValTok{1}\NormalTok{,}\DecValTok{4}\NormalTok{));}
\NormalTok{\}}

\end{Highlighting}
\end{Shaded}

\subsubsection{Shared Ptr - Shared
Ownership}\label{shared-ptr---shared-ownership}

\begin{itemize}
\itemsep1pt\parskip0pt\parsep0pt
\item
  Many pointers pointing to same object
\item
  Object only deleted if no pointers refer to it
\item
  Achieved via reference counting
\end{itemize}

\subsubsection{Shared Ptr Control Block}\label{shared-ptr-control-block}

\begin{itemize}
\item
  Won't go to too many details: 
\item
  From
  \href{https://www.amazon.co.uk/Effective-Modern-Specific-Ways-Improve/dp/1491903996/ref=sr_1_1?ie=UTF8\&qid=1484571499\&sr=8-1\&keywords=Effective+Modern+C\%2B\%2B}{``Effective
  Modern C++'', Meyers, p140}
\end{itemize}

\subsubsection{Shared Ptr - Usage 1}\label{shared-ptr---usage-1}

\begin{itemize}
\itemsep1pt\parskip0pt\parsep0pt
\item
  Place raw pointer straight into shared\_ptr
\item
  Pass to functions, reference or by value
\item
  Copy/Move constructors and assignment all implemented
\end{itemize}

\subsubsection{Shared Ptr - Usage 2}\label{shared-ptr---usage-2}

\begin{Shaded}
\begin{Highlighting}[]
\OtherTok{#include "Fraction.h"}
\OtherTok{#include <memory>}
\OtherTok{#include <iostream>}
\DataTypeTok{void} \NormalTok{divideBy2(}\DataTypeTok{const} \NormalTok{std::shared_ptr<Fraction>& f)}
\NormalTok{\{}
  \NormalTok{f->denominator *= }\DecValTok{2}\NormalTok{;}
\NormalTok{\}}
\DataTypeTok{void} \NormalTok{multiplyBy2(}\DataTypeTok{const} \NormalTok{std::shared_ptr<Fraction> f)}
\NormalTok{\{}
  \NormalTok{f->numerator *= }\DecValTok{2}\NormalTok{;}
\NormalTok{\}}
\DataTypeTok{int} \NormalTok{main() \{}
  \NormalTok{std::shared_ptr<Fraction> f1(}\KeywordTok{new} \NormalTok{Fraction(}\DecValTok{1}\NormalTok{,}\DecValTok{4}\NormalTok{));}
  \NormalTok{std::shared_ptr<Fraction> f2 = f1;}
  \NormalTok{divideBy2(f1);}
  \NormalTok{multiplyBy2(f2);}
  \NormalTok{std::cout << }\StringTok{"Value="} \NormalTok{<< f1->numerator << }\StringTok{"/"} \NormalTok{<< f1->denominator << std::endl;}
  \NormalTok{std::cout << }\StringTok{"f1="} \NormalTok{<< f1.get() << }\StringTok{", f2="} \NormalTok{<< f2.get() << std::endl;}
\NormalTok{\}}

\end{Highlighting}
\end{Shaded}

\subsubsection{Shared Ptr - Usage 3}\label{shared-ptr---usage-3}

\begin{itemize}
\itemsep1pt\parskip0pt\parsep0pt
\item
  Watch out for exceptions.
\item
  \href{https://www.amazon.co.uk/Effective-Modern-Specific-Ways-Improve/dp/1491903996/ref=sr_1_1?ie=UTF8\&qid=1484571499\&sr=8-1\&keywords=Effective+Modern+C\%2B\%2B}{``Effective
  Modern C++'', Meyers, p140}
\end{itemize}

\begin{Shaded}
\begin{Highlighting}[]
\OtherTok{#include "Fraction.h"}
\OtherTok{#include <memory>}
\OtherTok{#include <stdexcept>}
\OtherTok{#include <vector>}
\DataTypeTok{int} \NormalTok{checkSomething(}\DataTypeTok{const} \NormalTok{std::shared_ptr<Fraction>& f, }\DataTypeTok{const} \DataTypeTok{int}\NormalTok{& i)}
\NormalTok{\{}
  \CommentTok{// whatever.}
\NormalTok{\}}
\DataTypeTok{int} \NormalTok{computeSomethingFirst()}
\NormalTok{\{}
  \CommentTok{// what if this throws?}
\NormalTok{\}}
\DataTypeTok{int} \NormalTok{main()}
\NormalTok{\{}
  \NormalTok{std::vector<std::shared_ptr<Fraction> >  spaceForLotsOfFractions;}
  \DataTypeTok{int} \NormalTok{result = checkSomething(std::shared_ptr<Fraction>(}\KeywordTok{new} \NormalTok{Fraction(}\DecValTok{1}\NormalTok{,}\DecValTok{4}\NormalTok{)),}
                              \NormalTok{computeSomethingFirst()}
                             \NormalTok{);}
\NormalTok{\}}
\end{Highlighting}
\end{Shaded}

\subsubsection{Shared Ptr - Usage 4}\label{shared-ptr---usage-4}

\begin{itemize}
\itemsep1pt\parskip0pt\parsep0pt
\item
  Prefer \texttt{std::make\_shared}
\item
  Exception safe
\end{itemize}

\begin{Shaded}
\begin{Highlighting}[]
\OtherTok{#include "Fraction.h"}
\OtherTok{#include <memory>}
\OtherTok{#include <stdexcept>}
\OtherTok{#include <vector>}
\DataTypeTok{int} \NormalTok{checkSomething(}\DataTypeTok{const} \NormalTok{std::shared_ptr<Fraction>& f, }\DataTypeTok{const} \DataTypeTok{int}\NormalTok{& i)}
\NormalTok{\{}
  \CommentTok{// whatever.}
\NormalTok{\}}
\DataTypeTok{int} \NormalTok{computeSomethingFirst()}
\NormalTok{\{}
  \CommentTok{// what if this throws?}
\NormalTok{\}}
\DataTypeTok{int} \NormalTok{main()}
\NormalTok{\{}
  \NormalTok{std::vector<std::shared_ptr<Fraction> >  spaceForLotsOfFractions;}
  \DataTypeTok{int} \NormalTok{result = checkSomething(std::make_shared<Fraction>(}\DecValTok{1}\NormalTok{,}\DecValTok{4}\NormalTok{),}
                              \NormalTok{computeSomethingFirst()}
                             \NormalTok{);}
\NormalTok{\}}
\end{Highlighting}
\end{Shaded}

\subsubsection{Weak Ptr - Why?}\label{weak-ptr---why}

\begin{itemize}
\itemsep1pt\parskip0pt\parsep0pt
\item
  Like a shared pointer, but doesn't actually own anything
\item
  Use for example:

  \begin{itemize}
  \itemsep1pt\parskip0pt\parsep0pt
  \item
    Caches
  \item
    Break circular pointers
  \end{itemize}
\item
  Limited API
\item
  Not terribly common as most code ends up as hierarchies
\end{itemize}

\subsubsection{Weak Ptr - Example}\label{weak-ptr---example}

\begin{itemize}
\itemsep1pt\parskip0pt\parsep0pt
\item
  See
  \href{http://www.umich.edu/~eecs381/handouts/C++11_smart_ptrs.pdf}{David
  Kieras online paper}
\end{itemize}

\begin{Shaded}
\begin{Highlighting}[]
\OtherTok{#include "Fraction.h"}
\OtherTok{#include <memory>}
\OtherTok{#include <iostream>}
\DataTypeTok{int} \NormalTok{main() \{}
  \NormalTok{std::shared_ptr<Fraction> s1(}\KeywordTok{new} \NormalTok{Fraction(}\DecValTok{1}\NormalTok{,}\DecValTok{4}\NormalTok{));}
  \NormalTok{std::weak_ptr<Fraction> w1;      }\CommentTok{// can point to nothing}
  \NormalTok{std::weak_ptr<Fraction> w2 = s1; }\CommentTok{// assignment from shared}
  \NormalTok{std::weak_ptr<Fraction> w3(s1);  }\CommentTok{// construction from shared}

  \CommentTok{// Can't be de-referenced!!!}
  \CommentTok{// std::cerr << "Value=" << w1->numerator << "/" << w1->denominator << std::endl;}

  \CommentTok{// Needs converting to shared, and checking}
  \NormalTok{std::shared_ptr<Fraction> s2 = w1.lock();}
  \KeywordTok{if} \NormalTok{(s2)}
  \NormalTok{\{}
    \NormalTok{std::cout << }\StringTok{"Object w1 exists="} \NormalTok{<< s2->numerator << }\StringTok{"/"} \NormalTok{<< s2->denominator << std::endl;}
  \NormalTok{\}}

  \CommentTok{// Or, create shared, check for exception}
  \NormalTok{std::shared_ptr<Fraction> s3(w2);}
  \NormalTok{std::cout << }\StringTok{"Object must exists="} \NormalTok{<< s3->numerator << }\StringTok{"/"} \NormalTok{<< s3->denominator << std::endl;}
\NormalTok{\}}

\end{Highlighting}
\end{Shaded}

\subsubsection{Final Advice}\label{final-advice}

\begin{itemize}
\itemsep1pt\parskip0pt\parsep0pt
\item
  Benefits of immediate, fine-grained, garbage collection
\item
  Just ask
  \href{https://www.amazon.co.uk/Effective-Modern-Specific-Ways-Improve/dp/1491903996/ref=sr_1_1?ie=UTF8\&qid=1484571499\&sr=8-1\&keywords=Effective+Modern+C\%2B\%2B}{Scott
  Meyers!}

  \begin{itemize}
  \itemsep1pt\parskip0pt\parsep0pt
  \item
    Use \texttt{unique\_ptr} for unique ownership
  \item
    Easy to convert \texttt{unique\_ptr} to \texttt{shared\_ptr}
  \item
    But not the reverse
  \item
    Use \texttt{shared\_ptr} for shared resource management
  \item
    Avoid raw \texttt{new} - use \texttt{make\_shared},
    \texttt{make\_unique}
  \item
    Use \texttt{weak\_ptr} for pointers that can dangle (cache etc)
  \end{itemize}
\end{itemize}

\subsubsection{Conclusion for Smart
Pointers}\label{conclusion-for-smart-pointers}

\begin{itemize}
\itemsep1pt\parskip0pt\parsep0pt
\item
  Default to standard library, check compiler
\item
  Lots of other Smart Pointers

  \begin{itemize}
  \itemsep1pt\parskip0pt\parsep0pt
  \item
    \href{http://www.boost.org}{Boost} (use STL).
  \item
    \href{http://www.itk.org}{ITK}
  \item
    \href{http://www.vtk.org/Wiki/VTK/Tutorials/SmartPointers}{VTK}
  \item
    \href{https://wiki.qt.io/Smart_Pointers}{Qt Smart Pointers}
  \end{itemize}
\item
  Don't be tempted to write your own
\item
  Always read the manual
\item
  Always consistently use it
\end{itemize}

\subsubsection{Homework - 14}\label{homework---14}

\begin{itemize}
\itemsep1pt\parskip0pt\parsep0pt
\item
  Create a memory leak using bare pointers and \texttt{new} to heap
  allocate and then repeat using using smart pointers
\item
  Think of a use case where a resource should be uniquely owned and
  implement as a short application
\item
  Think of a use case where the resource is shared between multiple and
  implement
\item
  \emph{Note: in lecture 6 you will learn how to use tools such as
  \texttt{valgrind} to consistently check for memory leaks}
\end{itemize}

\subsection{Lambda expressions}\label{lambda-expressions}

\subsubsection{Game changer for C++}\label{game-changer-for-c}

\begin{itemize}
\item
  Lambda expressions allow you to create an unnamed function object
  within code
\item
  Expression can be defined as they are passed to a function as an
  argument
\item
  Useful for varions STL \texttt{\_if} and comparison functions:
  \texttt{std::find\_if}, \texttt{std::remove\_if}, \texttt{std::sort},
  \texttt{std::lower\_bound} etc
\item
  Can be used for a one off call for a context specific function
\item
  From ``Effective Modern C++'', Meyers, p215
\item
  A game changer for C++ despite bringing no new expressive power to the
  language
\item
  Everything you can do with a lambda could be done by hand with a bit
  more typing
\item
  But the impact on day to day C++ software development is enormous\\
\item
  Allow expressions to be defined as they are being passed as an
  argument to a function
\end{itemize}

\begin{Shaded}
\begin{Highlighting}[]
   \NormalTok{std::find_if(container.begin(), container.end(),}
                \NormalTok{[](}\DataTypeTok{int} \NormalTok{val) \{ }\KeywordTok{return} \DecValTok{0} \NormalTok{< val && val < }\DecValTok{10}\NormalTok{; \});}
\end{Highlighting}
\end{Shaded}

\subsubsection{Basic syntax}\label{basic-syntax}

\begin{itemize}
\item
  \texttt{{[} captures {]} \{ body \};}
\item
  \texttt{captures} is comma separated list of variable from enclosing
  scope that the lambda can use
\item
  \texttt{body} is where the function is defined
\item
  \texttt{{[}x,y{]} \{ return x + y; \}}
\item
  Captures can be by value \texttt{{[}x{]}} or by reference
  \texttt{{[}\&x{]}}
\item
  \texttt{{[}={]}} and `{[}\&{]} are default capture modes for all
  variables in enclosing scope -\textgreater{} discouraged as can lead
  to dangling references and are not thread safe (``Effective Modern
  C++'', Meyers, p216)
\item
  \texttt{{[} captures {]} ( params ) \{ body \};}
\item
  \texttt{params} list of params as with named functions except cannot
  have default value
\item
  \texttt{{[}{]} (int x, int y) \{ return x*x + y*y; \}(1,2)} would
  return \texttt{5}
\item
  \texttt{{[} captures {]} ( params ) -\textgreater{} ret \{ body \};}
\item
  \texttt{ret} is the return type, if not specified inferred from the
  return statement in the function
\item
  \texttt{{[}{]} () -\textgreater{} float \{ return "a"; \}} would give
  \texttt{error: cannot convert const char* to float in return}
\item
  It is possible to copy and reuse lambdas
\end{itemize}

\begin{verbatim}
  auto c1 = [](int y) { return y*y; };
  auto c2 = c1; // c2 is a copy of c1
  cout << "c1(2) = " << c1(2) << endl;
  cout << "c2(4) = " << c2(4) << endl;
\end{verbatim}

\begin{itemize}
\itemsep1pt\parskip0pt\parsep0pt
\item
  Gives
\end{itemize}

\begin{Shaded}
\begin{Highlighting}[]
\NormalTok{c1(}\DecValTok{2}\NormalTok{) = }\DecValTok{4}
\NormalTok{c2(}\DecValTok{4}\NormalTok{) = }\DecValTok{16}
\end{Highlighting}
\end{Shaded}

\subsubsection{Example use}\label{example-use}

\begin{Shaded}
\begin{Highlighting}[]
  \NormalTok{std::vector<}\DataTypeTok{int}\NormalTok{> v \{ }\DecValTok{1}\NormalTok{,}\DecValTok{2}\NormalTok{,}\DecValTok{3}\NormalTok{,}\DecValTok{4}\NormalTok{,}\DecValTok{5}\NormalTok{,}\DecValTok{6}\NormalTok{,}\DecValTok{7}\NormalTok{,}\DecValTok{8}\NormalTok{,}\DecValTok{9}\NormalTok{,}\DecValTok{10} \NormalTok{\};}
  \DataTypeTok{int} \NormalTok{neven = std::count_if(v.begin(), v.end(), [](}\DataTypeTok{int} \NormalTok{i)\{ }\KeywordTok{return} \NormalTok{i % }\DecValTok{2} \NormalTok{== }\DecValTok{0}\NormalTok{; \});}
\end{Highlighting}
\end{Shaded}

\subsubsection{Homework 15}\label{homework-15}

\begin{itemize}
\itemsep1pt\parskip0pt\parsep0pt
\item
  Create your own lambda expressions for each of the three basic syntax
  examples given above
\item
  Try to change a param from within, can you see a different behaviour
  if passed by reference or by value?
\item
  Use \texttt{std::count\_if} with an appropriate lambda expression to
  count the number of values in a
  \texttt{vector\textless{}int\textgreater{}} that are divisable by 2 or
  3
\end{itemize}

\subsubsection{Homework 16}\label{homework-16}

\begin{itemize}
\itemsep1pt\parskip0pt\parsep0pt
\item
  Create a simple \texttt{Student} class that has public member
  variables storing \texttt{string firstname},
  \texttt{string secondname} and \texttt{int age}
\item
  Create a vector and fill it with various instances of the
  \texttt{Student} class
\item
  Create a sorting class \texttt{StudentSort} and that has a method
  \texttt{vector\textless{}Student\textgreater{} SortByAge(vector\textless{}Student\textgreater{} vs);}
  that returns a \texttt{vector\textless{}Student\textgreater{}} that
  has been sorted by age
\item
  Use a \texttt{std::sort} and a lambda expression for this
\item
  Add a \texttt{bool} switch to
  \texttt{SortByAge(vector\textless{}Student\textgreater{} vs, bool reverse = false)}
  that reverses the sort
\end{itemize}

\subsection{Error Handling}\label{error-handling}

\subsubsection{Exceptions}\label{exceptions}

\begin{itemize}
\itemsep1pt\parskip0pt\parsep0pt
\item
  Exceptions are the C++ or Object Oriented way of Error Handling
\end{itemize}

\subsubsection{Exception Handling
Example}\label{exception-handling-example}

\begin{Shaded}
\begin{Highlighting}[]
\OtherTok{#include <stdexcept>}
\OtherTok{#include <iostream>}
\DataTypeTok{bool} \NormalTok{someFunction() \{ }\KeywordTok{return} \KeywordTok{false}\NormalTok{; \}}

\DataTypeTok{int} \NormalTok{main()}
\NormalTok{\{}
  \KeywordTok{try}
  \NormalTok{\{}
    \DataTypeTok{bool} \NormalTok{isOK = }\KeywordTok{false}\NormalTok{;}
    \NormalTok{isOK = someFunction();}
    \KeywordTok{if} \NormalTok{(!isOK)}
    \NormalTok{\{}
      \KeywordTok{throw} \NormalTok{std::runtime_error(}\StringTok{"Something is wrong"}\NormalTok{);}
    \NormalTok{\}}
  \NormalTok{\}}
  \KeywordTok{catch} \NormalTok{(std::exception& e)}
  \NormalTok{\{}
    \NormalTok{std::cerr << }\StringTok{"Caught Exception:"} \NormalTok{<< e.what() << std::endl;}
  \NormalTok{\}}
\NormalTok{\}}

\end{Highlighting}
\end{Shaded}

\subsubsection{What's the Point?}\label{whats-the-point}

\begin{itemize}
\itemsep1pt\parskip0pt\parsep0pt
\item
  A good summary
  \href{https://msdn.microsoft.com/en-us/library/hh279678.aspx}{here}:

  \begin{itemize}
  \itemsep1pt\parskip0pt\parsep0pt
  \item
    Have separated error handling logic from application logic
  \item
    Forces calling code to recognize an error condition and handle it
  \item
    Stack-unwinding destroys all objects in scope according to
    well-defined rules
  \item
    A clean separation between code that detects error and code that
    handles error
  \end{itemize}
\item
  First, lets look at C-style return codes
\end{itemize}

\subsubsection{Error Handling C-Style}\label{error-handling-c-style}

\begin{Shaded}
\begin{Highlighting}[]
\DataTypeTok{int} \NormalTok{foo(}\DataTypeTok{int} \NormalTok{a, }\DataTypeTok{int} \NormalTok{b)}
\NormalTok{\{}
  \CommentTok{// stuff}
  
  \KeywordTok{if}\NormalTok{(some error condition)}
  \NormalTok{\{}
    \KeywordTok{return} \DecValTok{1}\NormalTok{;}
  \NormalTok{\} }\KeywordTok{else} \KeywordTok{if} \NormalTok{(another error condition) \{}
    \KeywordTok{return} \DecValTok{2}\NormalTok{;}
  \NormalTok{\} }\KeywordTok{else} \NormalTok{\{}
    \KeywordTok{return} \DecValTok{0}\NormalTok{;}
  \NormalTok{\}}
\NormalTok{\}}

\DataTypeTok{void} \NormalTok{caller(}\DataTypeTok{int} \NormalTok{a, }\DataTypeTok{int} \NormalTok{b) }
\NormalTok{\{}
 \DataTypeTok{int} \NormalTok{result = foo(a, b);}
 \KeywordTok{if} \NormalTok{(result == }\DecValTok{1}\NormalTok{) }\CommentTok{// do something}
 \KeywordTok{else} \KeywordTok{if} \NormalTok{(result == }\DecValTok{2}\NormalTok{) }\CommentTok{// do something difference}
 \KeywordTok{else} 
 \NormalTok{\{}
   \CommentTok{// All ok, continue as you wish}
 \NormalTok{\}}
\NormalTok{\} }
\end{Highlighting}
\end{Shaded}

\subsubsection{Outcome}\label{outcome}

\begin{itemize}
\itemsep1pt\parskip0pt\parsep0pt
\item
  Can be perfectly usable
\item
  Depends on depth of function call stack
\item
  Depends on complexity of program
\item
  If deep/large, then can become unweildy
\end{itemize}

\subsubsection{Error Handling C++ Style}\label{error-handling-c-style-1}

\begin{Shaded}
\begin{Highlighting}[]
\OtherTok{#include <stdexcept>}
\OtherTok{#include <iostream>}

\DataTypeTok{int} \NormalTok{ReadNumberFromFile(}\DataTypeTok{const} \NormalTok{std::string& fileName)}
\NormalTok{\{}
  \KeywordTok{if} \NormalTok{(fileName.length() == }\DecValTok{0}\NormalTok{)}
  \NormalTok{\{}
    \KeywordTok{throw} \NormalTok{std::runtime_error(}\StringTok{"Empty fileName provided"}\NormalTok{);}
  \NormalTok{\}}

  \CommentTok{// Check for file existence etc. throw io errors.}

  \CommentTok{// do stuff}
  \KeywordTok{return} \DecValTok{2}\NormalTok{; }\CommentTok{// returning dummy number to force error}
\NormalTok{\}}


\DataTypeTok{void} \NormalTok{ValidateNumber(}\DataTypeTok{int} \NormalTok{number)}
\NormalTok{\{}
  \KeywordTok{if} \NormalTok{(number < }\DecValTok{3}\NormalTok{)}
  \NormalTok{\{}
    \KeywordTok{throw} \NormalTok{std::logic_error(}\StringTok{"Number is < 3"}\NormalTok{);}
  \NormalTok{\}}
  \KeywordTok{if} \NormalTok{(number > }\DecValTok{10}\NormalTok{)}
  \NormalTok{\{}
    \KeywordTok{throw} \NormalTok{std::logic_error(}\StringTok{"Number is > 10"}\NormalTok{);}
  \NormalTok{\}}
\NormalTok{\}}


\DataTypeTok{int} \NormalTok{main(}\DataTypeTok{int} \NormalTok{argc, }\DataTypeTok{char}\NormalTok{** argv)}
\NormalTok{\{}
  \KeywordTok{try}
  \NormalTok{\{}
    \KeywordTok{if} \NormalTok{(argc < }\DecValTok{2}\NormalTok{)}
    \NormalTok{\{}
      \NormalTok{std::cerr << }\StringTok{"Usage: "} \NormalTok{<< argv[}\DecValTok{0}\NormalTok{] << }\StringTok{" fileName"} \NormalTok{<< std::endl;}
      \KeywordTok{return} \NormalTok{EXIT_FAILURE;}
    \NormalTok{\}}

    \DataTypeTok{int} \NormalTok{myNumber = ReadNumberFromFile(argv[}\DecValTok{1}\NormalTok{]);}
    \NormalTok{ValidateNumber(myNumber);}
    
    \CommentTok{// Compute stuff.}

    \KeywordTok{return} \NormalTok{EXIT_SUCCESS;}

  \NormalTok{\}}
  \KeywordTok{catch} \NormalTok{(std::exception& e)}
  \NormalTok{\{}
    \NormalTok{std::cerr << }\StringTok{"Caught Exception:"} \NormalTok{<< e.what() << std::endl;}
  \NormalTok{\}}
\NormalTok{\}}

\end{Highlighting}
\end{Shaded}

\subsubsection{Comments}\label{comments}

\begin{itemize}
\itemsep1pt\parskip0pt\parsep0pt
\item
  Code that throws does not worry about the catcher
\item
  Exceptions are classes, can carry data
\item
  More suited to larger libraries of re-usable functions
\item
  Many different catchers, all implementing different error handling
\item
  Generally scales better, more flexible
\end{itemize}

\subsubsection{Practical Tips For Exception
Handling}\label{practical-tips-for-exception-handling}

\begin{itemize}
\itemsep1pt\parskip0pt\parsep0pt
\item
  Decide on error handling strategy at start
\item
  Use it consistently
\item
  Create your own base class exception
\item
  Derive all your exceptions from that base class
\item
  Stick to a few obvious classes, not one class for every single error
\end{itemize}

\subsubsection{Homework 17}\label{homework-17}

\begin{itemize}
\itemsep1pt\parskip0pt\parsep0pt
\item
  Taking the \texttt{Fraction} class from homework 8:

  \begin{itemize}
  \itemsep1pt\parskip0pt\parsep0pt
  \item
    Try to call \texttt{simplify} for a fraction with a denominator of 0
    and see what exception is thrown

    \begin{itemize}
    \itemsep1pt\parskip0pt\parsep0pt
    \item
      Try to catch this exception from the calling code
    \end{itemize}
  \item
    Create your own exception class that is thrown instead. It should
    inherit from \texttt{std::exception} (see
    \href{https://en.cppreference.com/w/cpp/error/exception}{cppreference/error/exception})

    \begin{itemize}
    \itemsep1pt\parskip0pt\parsep0pt
    \item
      Catch this from the calling code
    \end{itemize}
  \item
    Create the fraction and call \texttt{simplify} from within a
    function that is called from the main calling code

    \begin{itemize}
    \itemsep1pt\parskip0pt\parsep0pt
    \item
      Check you can catch the exception either in the calling code or
      from within the function
    \end{itemize}
  \end{itemize}
\end{itemize}

\subsubsection{More Practical Tips For Exception
Handling}\label{more-practical-tips-for-exception-handling}

\begin{itemize}
\itemsep1pt\parskip0pt\parsep0pt
\item
  Look at \href{http://www.cplusplus.com/reference/exception/}{C++
  standard classes} and
  \href{http://www.cplusplus.com/doc/tutorial/exceptions/}{tutorial}
\item
  An exception macro may be useful, e.g.
  \href{https://github.com/MITK/MITK/blob/master/Modules/Core/include/mitkException.h}{mitk::Exception}
  and
  \href{https://github.com/MITK/MITK/blob/master/Modules/Core/include/mitkExceptionMacro.h}{mithThrow()}
\end{itemize}

\section{Lecture 4: Modern C++ (3)}\label{lecture-4-modern-c-3}

\subsection{Patterns and Templates}\label{patterns-and-templates}

\subsubsection{The Story So Far}\label{the-story-so-far-2}

\begin{itemize}
\itemsep1pt\parskip0pt\parsep0pt
\item
  Git, CMake, Catch2
\item
  Recap of Modern C++ features
\item
  OO concepts
\item
  C++ Standard Library
\item
  Smart pointers and move semantics
\item
  Lambda expressions
\item
  Error handling
\end{itemize}

\subsubsection{Todays Lesson}\label{todays-lesson-2}

\begin{itemize}
\itemsep1pt\parskip0pt\parsep0pt
\item
  How to assemble/organise classes

  \begin{itemize}
  \itemsep1pt\parskip0pt\parsep0pt
  \item
    Beginner mistakes

    \begin{itemize}
    \itemsep1pt\parskip0pt\parsep0pt
    \item
      1 class - all functionality
    \item
      Deep inheritance trees
    \end{itemize}
  \end{itemize}
\item
  Patterns

  \begin{itemize}
  \itemsep1pt\parskip0pt\parsep0pt
  \item
    RAII
  \item
    Common OO design patterns
  \end{itemize}
\item
  Templates
\item
  Function templates
\item
  Class templates
\item
  Template specialisation
\end{itemize}

\subsection{Program To Interfaces}\label{program-to-interfaces}

\subsubsection{Why?}\label{why}

\begin{itemize}
\itemsep1pt\parskip0pt\parsep0pt
\item
  In research code we often ``just start hacking''
\item
  You tend to mix interface and implementation
\item
  Results in client/user of a class having implicit dependency on the
  implementation
\item
  So, define a pure virtual class, not for inheritance, but for clean
  API
\end{itemize}

\subsubsection{Example}\label{example}

\begin{Shaded}
\begin{Highlighting}[]
\OtherTok{#include <memory>}
\OtherTok{#include <vector>}
\OtherTok{#include <string>}

\KeywordTok{class} \NormalTok{DataPlayerI \{}
\KeywordTok{public}\NormalTok{:}
  \KeywordTok{virtual} \DataTypeTok{void} \NormalTok{StartPlaying() = }\DecValTok{0}\NormalTok{;}
  \KeywordTok{virtual} \DataTypeTok{void} \NormalTok{StopPlaying() = }\DecValTok{0}\NormalTok{;}
\NormalTok{\};}

\KeywordTok{class} \NormalTok{FileDataPlayer : }\KeywordTok{public} \NormalTok{DataPlayerI \{}
\KeywordTok{public}\NormalTok{:}
  \NormalTok{FileDataPlayer(}\DataTypeTok{const} \NormalTok{std::string& fileName)\{\}; }\CommentTok{// opens file    (RAII)}
  \NormalTok{~FileDataPlayer()\{\};                           }\CommentTok{// releases file (RAII)}
\KeywordTok{public}\NormalTok{:}
  \KeywordTok{virtual} \DataTypeTok{void} \NormalTok{StartPlaying() \{\};}
  \KeywordTok{virtual} \DataTypeTok{void} \NormalTok{StopPlaying() \{\};}
\NormalTok{\};}

\KeywordTok{class} \NormalTok{Experiment \{}
\KeywordTok{public}\NormalTok{:}
  \NormalTok{Experiment(DataPlayerI *d) \{ m_Player.reset(d); \} }\CommentTok{// takes ownership}
  \DataTypeTok{void} \NormalTok{Run() \{\};}
  \NormalTok{std::vector<std::string> GetResults() }\DataTypeTok{const} \NormalTok{\{\};}
\KeywordTok{private}\NormalTok{:}
  \NormalTok{std::unique_ptr<DataPlayerI> m_Player;  }
\NormalTok{\};}

\DataTypeTok{int} \NormalTok{main(}\DataTypeTok{int} \NormalTok{argc, }\DataTypeTok{char}\NormalTok{** argv)}
\NormalTok{\{}
  \NormalTok{FileDataPlayer fdp(argv[}\DecValTok{1}\NormalTok{]); }\CommentTok{// Or some class WebDataPlayer derived from DataPlayerI}
  \NormalTok{Experiment e(&fdp);}
  \NormalTok{e.Run();}

  \CommentTok{// etc.}
\NormalTok{\}}
\end{Highlighting}
\end{Shaded}

\subsubsection{Comments}\label{comments-1}

\begin{itemize}
\itemsep1pt\parskip0pt\parsep0pt
\item
  Useful between sub-components of a system

  \begin{itemize}
  \itemsep1pt\parskip0pt\parsep0pt
  \item
    GUI front end, Web back end
  \item
    Logic and Database
  \end{itemize}
\item
  Is useful in general to force loose connections between areas of code

  \begin{itemize}
  \itemsep1pt\parskip0pt\parsep0pt
  \item
    e.g.~different libraries that have different dependencies
  \item
    define an interface that just exports standard types
  \item
    stops the spread of dependencies
  \item
    Lookup \href{http://en.cppreference.com/w/cpp/language/pimpl}{Pimpl}
    idiom
  \end{itemize}
\end{itemize}

\subsection{Inheritance}\label{inheritance-1}

\subsubsection{Don't overuse
Inheritance}\label{dont-overuse-inheritance}

\begin{itemize}
\itemsep1pt\parskip0pt\parsep0pt
\item
  Inheritance is not just for saving duplication
\item
  It MUST represent derived/related types
\item
  Derived class must truely represent `is-a' relationship
\item
  eg `Square' is-a `Shape'
\item
  Deep inheritance hierarchies are almost always wrong
\item
  If something `doesn't quite fit' check your inheritance
\end{itemize}

\subsubsection{Surely Its Simple?}\label{surely-its-simple}

\begin{itemize}
\itemsep1pt\parskip0pt\parsep0pt
\item
  Common example: Square/Rectangle problem,
  \href{http://www.oodesign.com/liskov-s-substitution-principle.html}{here}
\end{itemize}

\begin{Shaded}
\begin{Highlighting}[]
\OtherTok{#include <iostream>}

\KeywordTok{class} \NormalTok{Rectangle \{}
\KeywordTok{public}\NormalTok{:}
  \NormalTok{Rectangle() : m_Width(}\DecValTok{0}\NormalTok{), m_Height(}\DecValTok{0}\NormalTok{) \{\};}
  \KeywordTok{virtual} \NormalTok{~Rectangle()\{\};}
  \DataTypeTok{int} \NormalTok{GetArea() }\DataTypeTok{const} \NormalTok{\{ }\KeywordTok{return} \NormalTok{m_Width*m_Height; \}}
  \KeywordTok{virtual} \DataTypeTok{void} \NormalTok{SetWidth(}\DataTypeTok{int} \NormalTok{w) \{ m_Width=w; \}}
  \KeywordTok{virtual} \DataTypeTok{void} \NormalTok{SetHeight(}\DataTypeTok{int} \NormalTok{h) \{ m_Height=h; \}}
\KeywordTok{protected}\NormalTok{:}
  \DataTypeTok{int} \NormalTok{m_Width;}
  \DataTypeTok{int} \NormalTok{m_Height;}
\NormalTok{\};}

\KeywordTok{class} \NormalTok{Square : }\KeywordTok{public} \NormalTok{Rectangle \{}
\KeywordTok{public}\NormalTok{:}
  \NormalTok{Square()\{\};}
  \NormalTok{~Square()\{\};}
  \KeywordTok{virtual} \DataTypeTok{void} \NormalTok{SetWidth(}\DataTypeTok{int} \NormalTok{w) \{ m_Width=w; m_Height=w; \}}
  \KeywordTok{virtual} \DataTypeTok{void} \NormalTok{SetHeight(}\DataTypeTok{int} \NormalTok{h) \{ m_Width=h; m_Height=h; \}}
\NormalTok{\};}

\DataTypeTok{int} \NormalTok{main() }
\NormalTok{\{}
  \NormalTok{Rectangle *r = }\KeywordTok{new} \NormalTok{Square();}
  \NormalTok{r->SetWidth(}\DecValTok{5}\NormalTok{);}
  \NormalTok{r->SetHeight(}\DecValTok{6}\NormalTok{);}
  \NormalTok{std::cout << }\StringTok{"Area = "} \NormalTok{<< r->GetArea() << std::endl;}
\NormalTok{\}}
\end{Highlighting}
\end{Shaded}

\subsubsection{Liskov Substitution
Principal}\label{liskov-substitution-principal}

\begin{itemize}
\itemsep1pt\parskip0pt\parsep0pt
\item
  \href{https://en.wikipedia.org/wiki/Liskov_substitution_principle}{Wikipedia}
\item
  ``if S is a subtype of T, then objects of type T may be replaced with
  objects of type S without altering any of the desirable properties of
  that program''
\item
  i.e.~is S is derived class from base class T then objects of type T
  should be able to be replaced with objects of type S
\item
  Can something truely be substituted?
\item
  If someone else filled a vector of type T, would I care what type I
  have?
\end{itemize}

\subsubsection{What to Look For}\label{what-to-look-for}

\begin{itemize}
\itemsep1pt\parskip0pt\parsep0pt
\item
  If you have:

  \begin{itemize}
  \itemsep1pt\parskip0pt\parsep0pt
  \item
    Methods you don't want to implement in derived class
  \item
    Methods that don't make sense in derived class
  \item
    Methods that are unneeded in derived class
  \item
    If you have a list of something, and someone else swapping in a
    derived type would cause problems
  \end{itemize}
\item
  Then you have probably got your inheritance wrong
\end{itemize}

\subsubsection{Composition Vs
Inheritance}\label{composition-vs-inheritance}

\begin{itemize}
\itemsep1pt\parskip0pt\parsep0pt
\item
  Lots of Info online eg.
  \href{https://en.wikipedia.org/wiki/Composition_over_inheritance}{wikipedia}
\item
  In basic OO Principals

  \begin{itemize}
  \itemsep1pt\parskip0pt\parsep0pt
  \item
    `Has-a' means `pointer or reference to'
  \item
    eg.\texttt{Car} has-a \texttt{Engine}
  \end{itemize}
\item
  But there is also:
\item
  \href{https://en.wikipedia.org/wiki/Object_composition\#Composition}{Composition}:
  Strong `has-a'. Component parts are owned by thing pointing to them.
\item
  \href{https://en.wikipedia.org/wiki/Object_composition\#Aggregation}{Aggregation}:
  Weak `has-a'. Component part has its own lifecycle.
\item
  Association: General term, referring to either composition or
  aggregation, just a `pointer-to'
\end{itemize}

\subsubsection{Examples}\label{examples}

\begin{itemize}
\itemsep1pt\parskip0pt\parsep0pt
\item
  House `has-a' set of Rooms. Destroying House, you destroy all Room
  objects. No point having a House on its own.
\item
  Room `has-a' Computer. If room is being refurbished, Computer should
  not be thrown away. Can go to another room
\end{itemize}

\subsubsection{But Why?}\label{but-why}

\begin{itemize}
\itemsep1pt\parskip0pt\parsep0pt
\item
  Good article:
  \href{https://www.thoughtworks.com/insights/blog/composition-vs-inheritance-how-choose}{Choosing
  Composition or Inheritance}
\item
  Composition is more flexible
\item
  Inheritance has much tighter definition than you realise
\end{itemize}

\subsection{Dependency Injection}\label{dependency-injection}

\subsubsection{Construction}\label{construction}

\begin{itemize}
\itemsep1pt\parskip0pt\parsep0pt
\item
  What could be wrong with this:
\end{itemize}

\begin{Shaded}
\begin{Highlighting}[]
\OtherTok{#include <memory>}

\KeywordTok{class} \NormalTok{Bar \{}
\NormalTok{\};}

\KeywordTok{class} \NormalTok{Foo \{}
\KeywordTok{public}\NormalTok{:}
  \NormalTok{Foo()}
  \NormalTok{\{}
    \NormalTok{m_Bar = }\KeywordTok{new} \NormalTok{Bar();}
  \NormalTok{\}}

\KeywordTok{private}\NormalTok{:}
  \NormalTok{Bar* m_Bar;}
\NormalTok{\};}

\DataTypeTok{int} \NormalTok{main()}
\NormalTok{\{}
  \NormalTok{Foo a;}
\NormalTok{\}}

\end{Highlighting}
\end{Shaded}

\subsubsection{Unwanted Dependencies}\label{unwanted-dependencies}

\begin{itemize}
\itemsep1pt\parskip0pt\parsep0pt
\item
  If constructor instantiates class directly:

  \begin{itemize}
  \itemsep1pt\parskip0pt\parsep0pt
  \item
    Hard-coded class name
  \item
    Duplication of initialisation code
  \end{itemize}
\end{itemize}

\subsubsection{Dependency Injection}\label{dependency-injection-1}

\begin{itemize}
\itemsep1pt\parskip0pt\parsep0pt
\item
  Read Martin Fowler's
  \href{http://www.martinfowler.com/articles/injection.html}{Inversion
  of Control Containers and the Dependency Injection Pattern}
\item
  Type 2 - Constructor Injection
\item
  Type 3 - Setter Injection
\end{itemize}

\subsubsection{Constructor Injection
Example}\label{constructor-injection-example}

\begin{Shaded}
\begin{Highlighting}[]
\OtherTok{#include <memory>}

\KeywordTok{class} \NormalTok{Bar \{}
\NormalTok{\};}

\KeywordTok{class} \NormalTok{Foo \{}
\KeywordTok{public}\NormalTok{:}
  \NormalTok{Foo(Bar* b)}
  \NormalTok{: m_Bar(b)}
  \NormalTok{\{}
  \NormalTok{\}}

\KeywordTok{private}\NormalTok{:}
  \NormalTok{Bar* m_Bar;}
\NormalTok{\};}

\DataTypeTok{int} \NormalTok{main()}
\NormalTok{\{}
  \NormalTok{Bar b;}
  \NormalTok{Foo a(&b);}
\NormalTok{\}}

\end{Highlighting}
\end{Shaded}

\subsubsection{Setter Injection Example}\label{setter-injection-example}

\begin{Shaded}
\begin{Highlighting}[]
\OtherTok{#include <memory>}

\KeywordTok{class} \NormalTok{Bar \{}
\NormalTok{\};}

\KeywordTok{class} \NormalTok{Foo \{}
\KeywordTok{public}\NormalTok{:}
  \NormalTok{Foo()}
  \NormalTok{\{}
  \NormalTok{\}}
  \DataTypeTok{void} \NormalTok{SetBar(Bar *b) \{ m_Bar = b; \}}

\KeywordTok{private}\NormalTok{:}
  \NormalTok{Bar* m_Bar;}
\NormalTok{\};}

\DataTypeTok{int} \NormalTok{main()}
\NormalTok{\{}
  \NormalTok{Bar b;}
  \NormalTok{Foo a();}
  \NormalTok{a.SetBar(&b);}
\NormalTok{\}}

\end{Highlighting}
\end{Shaded}

Question: Which is better?

\subsubsection{Advantages of Dependency
Injection}\label{advantages-of-dependency-injection}

\begin{itemize}
\itemsep1pt\parskip0pt\parsep0pt
\item
  Using Dependency Injection

  \begin{itemize}
  \itemsep1pt\parskip0pt\parsep0pt
  \item
    Removes hard coding of \texttt{new ClassName}
  \item
    Creation is done outside class, so class only uses public API
  \item
    Leads towards fewer assumptions in the code
  \end{itemize}
\end{itemize}

\subsubsection{Homework 18}\label{homework-18}

\begin{itemize}
\itemsep1pt\parskip0pt\parsep0pt
\item
  Taking the \texttt{Student} class from homework 16:
\item
  Create a new \texttt{Laptop} class that has a string \texttt{os} data
  member for the operating system name and an integer \texttt{year} data
  member for the year produced.
\item
  \texttt{Laptop} should have both a default constructor that sets
  \texttt{year} to 0 and name to ``Not set'' as well as an overloaded
  constructor that initialises both \texttt{year} and \texttt{os}
\item
  Modify \texttt{Student} to have a \texttt{Laptop} data member
\item
  Try out the two types of dependency injection above: constructor,
  setter
\item
  Confirm that the \texttt{Student} class is now invarient to changes in
  how you instantiate \texttt{Laptop}
\end{itemize}

\subsection{RAII Pattern}\label{raii-pattern}

\subsubsection{What is it?}\label{what-is-it-1}

\begin{itemize}
\itemsep1pt\parskip0pt\parsep0pt
\item
  \href{https://en.wikipedia.org/wiki/Resource_Acquisition_Is_Initialization}{Resource
  Allocation Is Initialisation (RAII)}
\item
  Obtain all resources in constructor
\item
  Release them all in destructor
\end{itemize}

\subsubsection{Why is it?}\label{why-is-it}

\begin{itemize}
\itemsep1pt\parskip0pt\parsep0pt
\item
  Guaranteed fully initialised object once constructor is complete
\item
  Objects on stack are guaranteed to be destroyed when an exception is
  thrown and stack is unwound

  \begin{itemize}
  \itemsep1pt\parskip0pt\parsep0pt
  \item
    Including smart pointers to objects
  \end{itemize}
\end{itemize}

\subsubsection{Example}\label{example-1}

\begin{itemize}
\itemsep1pt\parskip0pt\parsep0pt
\item
  You may already be using it:
  \href{https://en.wikipedia.org/wiki/Resource_Acquisition_Is_Initialization}{STL
  example}
\item
  \href{https://en.wikibooks.org/wiki/More_C\%2B\%2B_Idioms/Resource_Acquisition_Is_Initialization}{Another
  example}
\end{itemize}

\subsubsection{Homework 19}\label{homework-19}

\begin{itemize}
\itemsep1pt\parskip0pt\parsep0pt
\item
  Create a simple class \texttt{Foo} that contains a data member that is
  a raw pointer \texttt{bptr} to another class \texttt{Bar} that
  contains an integer as a data member
\item
  Add a \texttt{std::cout} to the constructor and destructor of both
  classes so that you know when they have been called
\item
  Implement the RAII pattern to create and destroy the \texttt{Bar}
  object that \texttt{bptr} points to
\item
  Create an instance of \texttt{Foo foo} in your application and confirm
  that if an exception is thrown before Foo goes out of scope that the
  destructor for both \texttt{Foo} and \texttt{Bar} are called and the
  \texttt{Bar} object is released
\end{itemize}

\subsection{Construction Patterns}\label{construction-patterns}

\subsubsection{Constructional Patterns}\label{constructional-patterns}

\begin{itemize}
\itemsep1pt\parskip0pt\parsep0pt
\item
  Other methods include

  \begin{itemize}
  \itemsep1pt\parskip0pt\parsep0pt
  \item
    See \href{https://en.wikipedia.org/wiki/Design_Patterns}{Gang of
    Four} book
  \item
    \href{https://en.wikipedia.org/wiki/Strategy_pattern}{Strategy
    Pattern}
  \item
    \href{https://en.wikipedia.org/wiki/Factory_method_pattern}{Factory
    Pattern}
  \item
    \href{https://en.wikipedia.org/wiki/Abstract_factory_pattern}{Abstract
    Factory Pattern}
  \item
    \href{https://en.wikipedia.org/wiki/Builder_pattern}{Builder
    Pattern}\\
  \end{itemize}
\item
  Also look up
  \href{https://en.wikipedia.org/wiki/Service_locator_pattern}{Service
  Locator Pattern}
\end{itemize}

\subsubsection{Managing Complexity}\label{managing-complexity}

\begin{itemize}
\itemsep1pt\parskip0pt\parsep0pt
\item
  Rather than monolithic code (bad)
\item
  We end up with many smaller classes (good)
\item
  So, its more flexible (good)
\item
  It does look more complex at first (don't panic)
\item
  You get used to thinking in small objects
\end{itemize}

\subsection{Using Templates}\label{using-templates}

\subsubsection{What Are Templates?}\label{what-are-templates}

\begin{itemize}
\itemsep1pt\parskip0pt\parsep0pt
\item
  C++ templates allow functions/classes to operate on generic types.
\item
  See: \href{http://en.wikipedia.org/wiki/Generic_programming}{Generic
  Programming}.
\item
  Write code, where `type' is provided later
\item
  Types instantiated at compile time, as they are needed
\item
  (Remember, C++ is strongly typed)
\end{itemize}

\subsubsection{You May Already Use
Them!}\label{you-may-already-use-them}

You probably use them already. Example type (class):

\begin{verbatim}
std::vector<int> myVectorInts;
\end{verbatim}

Example algorithm:
\href{http://www.cplusplus.com/reference/algorithm/sort/}{C++ sort}

\begin{verbatim}
std::sort(myVectorInts.begin(), myVectorInts.end());
\end{verbatim}

Aim: Write functions, classes, in terms of future/other/generic types,
type provided as parameter.

\subsubsection{Why Are Templates
Useful?}\label{why-are-templates-useful}

\begin{itemize}
\itemsep1pt\parskip0pt\parsep0pt
\item
  Generic programming:

  \begin{itemize}
  \itemsep1pt\parskip0pt\parsep0pt
  \item
    not pre-processor macros
  \item
    so maintain type safety
  \item
    separate algorithm from implementation
  \item
    extensible, optimisable via \href{97TemplateMetaProg}{Template
    Meta-Programming} (TMP)
  \end{itemize}
\end{itemize}

\subsubsection{Book}\label{book}

\begin{itemize}
\itemsep1pt\parskip0pt\parsep0pt
\item
  For more infomation see
  \href{http://erdani.com/index.php/books/modern-c-design/}{``Modern C++
  Design''}
\item
  2001, but still excellent text on templates, meta-programming, policy
  based design etc.
\item
  This section of course, gives basic introduction for research
  programmers
\end{itemize}

\subsubsection{Are Templates Difficult?}\label{are-templates-difficult}

\begin{itemize}
\itemsep1pt\parskip0pt\parsep0pt
\item
  Some say: notation is ugly

  \begin{itemize}
  \itemsep1pt\parskip0pt\parsep0pt
  \item
    Does take getting used to
  \item
    Use \texttt{typedef} to simplify
  \end{itemize}
\item
  Some say: verbose, confusing error messages

  \begin{itemize}
  \itemsep1pt\parskip0pt\parsep0pt
  \item
    Nothing intrinsically difficult
  \item
    Take small steps, compile regularly
  \item
    Learn to think like a compiler
  \end{itemize}
\item
  Code infiltration

  \begin{itemize}
  \itemsep1pt\parskip0pt\parsep0pt
  \item
    Use sparingly
  \item
    Hide usage behind clean interface
  \end{itemize}
\end{itemize}

\subsubsection{Why Templates in
Research?}\label{why-templates-in-research}

\begin{itemize}
\itemsep1pt\parskip0pt\parsep0pt
\item
  Minimise code duplication: think about how useful \texttt{std::vector}
  is
\item
  For example methods that generalise from 2D, 3D, n-dimensions, (e.g.
  \href{http://www.itk.org}{ITK} )
\item
  Test numerical code with simple types, apply to complex/other types
\end{itemize}

\subsubsection{Why Teach Templates?}\label{why-teach-templates}

\begin{itemize}
\itemsep1pt\parskip0pt\parsep0pt
\item
  Standard Template Library uses them
\item
  More common in research code, than business code
\item
  In research, more likely to `code for the unknown'
\item
  Boost, Qt, EIGEN uses them
\end{itemize}

\subsection{Function Templates}\label{function-templates}

\subsubsection{Function Templates
Example}\label{function-templates-example}

\begin{itemize}
\itemsep1pt\parskip0pt\parsep0pt
\item
  Credit to
  \href{http://www.cplusplus.com/doc/tutorial/functions2}{www.cplusplus.com}
\end{itemize}

\begin{Shaded}
\begin{Highlighting}[]
\CommentTok{// function template}
\OtherTok{#include <iostream>}
\KeywordTok{using} \KeywordTok{namespace} \NormalTok{std;}

\KeywordTok{template} \NormalTok{<}\KeywordTok{class} \NormalTok{T> }\CommentTok{// class|typename}
\NormalTok{T sum (T a, T b)}
\NormalTok{\{}
  \NormalTok{T result;}
  \NormalTok{result = a + b;}
  \KeywordTok{return} \NormalTok{result;}
\NormalTok{\}}

\DataTypeTok{int} \NormalTok{main () \{}
  \DataTypeTok{int} \NormalTok{i=}\DecValTok{5}\NormalTok{, j=}\DecValTok{6}\NormalTok{;}
  \DataTypeTok{double} \NormalTok{f=}\FloatTok{2.0}\NormalTok{, g=}\FloatTok{0.5}\NormalTok{;}
  \NormalTok{cout << sum<}\DataTypeTok{int}\NormalTok{>(i,j) << }\CharTok{'\textbackslash{}n'}\NormalTok{;}
  \NormalTok{cout << sum<}\DataTypeTok{double}\NormalTok{>(f,g) << }\CharTok{'\textbackslash{}n'}\NormalTok{;}
  \KeywordTok{return} \DecValTok{0}\NormalTok{;}
\NormalTok{\}}
\end{Highlighting}
\end{Shaded}

\begin{itemize}
\itemsep1pt\parskip0pt\parsep0pt
\item
  And produces this output when run
\end{itemize}

\begin{verbatim}
11
2.5
\end{verbatim}

\subsubsection{Why Use Function
Templates?}\label{why-use-function-templates}

\begin{itemize}
\itemsep1pt\parskip0pt\parsep0pt
\item
  Instead of function overloading

  \begin{itemize}
  \itemsep1pt\parskip0pt\parsep0pt
  \item
    Reduce your code duplication
  \item
    Reduce your maintenance
  \item
    Reduce your effort
  \item
    Also see this
    \href{http://www.codeproject.com/Articles/257589/An-Idiots-Guide-to-Cplusplus-Templates-Part}{Additional
    tutorial}.
  \end{itemize}
\end{itemize}

\subsubsection{Language Definition 1}\label{language-definition-1}

\begin{itemize}
\itemsep1pt\parskip0pt\parsep0pt
\item
  From the
  \href{http://en.cppreference.com/w/cpp/language/function_template}{language
  reference}
\end{itemize}

\begin{verbatim}
template < parameter-list > function-declaration
\end{verbatim}

\begin{itemize}
\itemsep1pt\parskip0pt\parsep0pt
\item
  so
\end{itemize}

\begin{verbatim}
template < class T >  // note 'class'
void MyFunction(T a, T b)
{
  // do something
}
\end{verbatim}

\begin{itemize}
\itemsep1pt\parskip0pt\parsep0pt
\item
  or
\end{itemize}

\begin{verbatim}
template < typename T1, typename T2 >  // note 'typename'
T1 MyFunctionTwoArgs(T1 a, T2 b)
{
  // do something
}
\end{verbatim}

\subsubsection{Language Definition 2}\label{language-definition-2}

\begin{itemize}
\itemsep1pt\parskip0pt\parsep0pt
\item
  Also

  \begin{itemize}
  \itemsep1pt\parskip0pt\parsep0pt
  \item
    Can use \texttt{class} or \texttt{typename}.
  \item
    I prefer \texttt{typename}.
  \item
    Template parameter can apply to references, pointers, return types,
    arrays etc.
  \end{itemize}
\end{itemize}

\subsubsection{Default Argument
Resolution}\label{default-argument-resolution}

\begin{itemize}
\itemsep1pt\parskip0pt\parsep0pt
\item
  Given:
\end{itemize}

\begin{verbatim}
double GetAverage<typename T>(const std::vector<T>& someNumbers);
\end{verbatim}

\begin{itemize}
\itemsep1pt\parskip0pt\parsep0pt
\item
  then:
\end{itemize}

\begin{verbatim}
std::vector<double> myNumbers;
double result = GetAverage(myNumbers);
\end{verbatim}

\begin{itemize}
\itemsep1pt\parskip0pt\parsep0pt
\item
  will call:
\end{itemize}

\begin{verbatim}
double GetAverage<double>(const std::vector<double>& someNumbers);
\end{verbatim}

\begin{itemize}
\itemsep1pt\parskip0pt\parsep0pt
\item
  So, if function parameters can inform the compiler uniquely as to
  which function to instantiate, its automatically compiled.
\end{itemize}

\subsubsection{Explicit Argument Resolution - part
1}\label{explicit-argument-resolution---part-1}

\begin{itemize}
\itemsep1pt\parskip0pt\parsep0pt
\item
  However, given:
\end{itemize}

\begin{verbatim}
double GetAverage<typename T>(const T& a, const T& b);
\end{verbatim}

\begin{itemize}
\itemsep1pt\parskip0pt\parsep0pt
\item
  and:
\end{itemize}

\begin{verbatim}
int a, b;
int result = GetAverage(a, b);
\end{verbatim}

\begin{itemize}
\itemsep1pt\parskip0pt\parsep0pt
\item
  But you don't want the int version called (due to integer division
  perhaps), you can:
\end{itemize}

\begin{verbatim}
double result = GetAverage<double>(a, b);
\end{verbatim}

\subsubsection{Explicit Argument Resolution - part
2}\label{explicit-argument-resolution---part-2}

\begin{itemize}
\itemsep1pt\parskip0pt\parsep0pt
\item
  equivalent to
\end{itemize}

\texttt{GetAverage\textless{}double\textgreater{}(static\_cast\textless{}double\textgreater{}(a), static\_cast\textless{}double\textgreater{}(b));}

\begin{itemize}
\item
  i.e.~name the template function parameter explicitly.
\item
  Cases for Explicit Template Argument Specification

  \begin{itemize}
  \itemsep1pt\parskip0pt\parsep0pt
  \item
    Force compilation of a specific version (eg. int as above)
  \item
    Also if method parameters do not allow compiler to deduce anything
    eg. \texttt{PrintSize()} method.
  \end{itemize}
\end{itemize}

\subsubsection{Beware of Code Bloat}\label{beware-of-code-bloat}

\begin{itemize}
\itemsep1pt\parskip0pt\parsep0pt
\item
  Given:
\end{itemize}

\begin{verbatim}
double GetMax<typename T1, typename T2>(const &T1, const &T2);
\end{verbatim}

\begin{itemize}
\itemsep1pt\parskip0pt\parsep0pt
\item
  and:
\end{itemize}

\begin{verbatim}
double r1 = GetMax(1,2);
double r2 = GetMax(1,2.0);
double r3 = GetMax(1.0,2.0);
\end{verbatim}

\begin{itemize}
\itemsep1pt\parskip0pt\parsep0pt
\item
  The compiler will generate 3 different max functions.
\item
  Be Careful

  \begin{itemize}
  \itemsep1pt\parskip0pt\parsep0pt
  \item
    Executables/libraries get larger
  \item
    Compilation time will increase
  \item
    Error messages get more verbose
  \end{itemize}
\end{itemize}

\subsubsection{Two Stage Compilation}\label{two-stage-compilation}

\begin{itemize}
\itemsep1pt\parskip0pt\parsep0pt
\item
  Basic syntax checking (eg. brackets, semi-colon, etc), when
  \texttt{\#include}'d
\item
  But only compiled when instantiated (eg. check existence of +
  operator).
\end{itemize}

\subsubsection{Instantiation}\label{instantiation}

\begin{itemize}
\itemsep1pt\parskip0pt\parsep0pt
\item
  Object Code is only really generated if code is used
\item
  Template functions can be

  \begin{itemize}
  \itemsep1pt\parskip0pt\parsep0pt
  \item
    .h file only
  \item
    .h file that includes separate .cxx/.txx/.hxx file (e.g.~ITK)
  \item
    .h file and separate .cxx/.txx file (sometimes by convention a .hpp
    file)
  \end{itemize}
\item
  In general

  \begin{itemize}
  \itemsep1pt\parskip0pt\parsep0pt
  \item
    Most libraries/people prefer header only implementations
  \end{itemize}
\end{itemize}

\subsubsection{Explicit Instantiation - part
1}\label{explicit-instantiation---part-1}

\begin{itemize}
\item
  Language Reference
  \href{http://en.cppreference.com/w/cpp/language/function_template}{here}
\item
  \href{http://msdn.microsoft.com/en-us/library/by56e477\%28VS.80\%29.aspx}{Microsoft
  Example}
\item
  Given (library) header:
\end{itemize}

\begin{Shaded}
\begin{Highlighting}[]
\OtherTok{#ifndef explicitInstantiation_h}
\OtherTok{#define explicitInstantiation_h}
\KeywordTok{template} \NormalTok{<}\KeywordTok{typename} \NormalTok{T> }\DataTypeTok{void} \NormalTok{f(T s);}
\OtherTok{#endif}
\end{Highlighting}
\end{Shaded}

\begin{itemize}
\itemsep1pt\parskip0pt\parsep0pt
\item
  Given (library) implementation:
\end{itemize}

\begin{Shaded}
\begin{Highlighting}[]
\OtherTok{#include <iostream>}
\OtherTok{#include <typeinfo>}
\OtherTok{#include "explicitInstantiation.h"}
\KeywordTok{template}\NormalTok{<}\KeywordTok{typename} \NormalTok{T>}
\DataTypeTok{void} \NormalTok{f(T s)}
\NormalTok{\{}
    \NormalTok{std::cout << }\KeywordTok{typeid}\NormalTok{(T).name() << }\StringTok{" "} \NormalTok{<< s << }\CharTok{'\textbackslash{}n'}\NormalTok{;}
\NormalTok{\}}
\KeywordTok{template} \DataTypeTok{void} \NormalTok{f<}\DataTypeTok{double}\NormalTok{>(}\DataTypeTok{double}\NormalTok{); }\CommentTok{// instantiates f<double>(double)}
\KeywordTok{template} \DataTypeTok{void} \NormalTok{f<>(}\DataTypeTok{char}\NormalTok{); }\CommentTok{// instantiates f<char>(char), template argument deduced}
\KeywordTok{template} \DataTypeTok{void} \NormalTok{f(}\DataTypeTok{int}\NormalTok{); }\CommentTok{// instantiates f<int>(int), template argument deduced}
\end{Highlighting}
\end{Shaded}

\subsubsection{Explicit Instantiation - part
2}\label{explicit-instantiation---part-2}

\begin{itemize}
\itemsep1pt\parskip0pt\parsep0pt
\item
  Given client code:
\end{itemize}

\begin{Shaded}
\begin{Highlighting}[]
\OtherTok{#include <iostream>}
\OtherTok{#include "explicitInstantiation.h"}

\DataTypeTok{int} \NormalTok{main(}\DataTypeTok{int} \NormalTok{argc, }\DataTypeTok{char}\NormalTok{** argv)}
\NormalTok{\{}
  \NormalTok{std::cout << }\StringTok{"Matt, double 1.0="} \NormalTok{<< std::endl;}
  \NormalTok{f(}\FloatTok{1.0}\NormalTok{);}
  \NormalTok{std::cout << }\StringTok{"Matt, char a="} \NormalTok{<< std::endl;}
  \NormalTok{f('a');}
  \NormalTok{std::cout << }\StringTok{"Matt, int 2="} \NormalTok{<< std::endl;}
  \NormalTok{f(}\DecValTok{2}\NormalTok{);}
\CommentTok{//  std::cout << "Matt, float 3.0=" << std::endl;}
\CommentTok{//  f<float>(static_cast<float>(3.0));  // compile error}
\NormalTok{\}}
\end{Highlighting}
\end{Shaded}

\begin{itemize}
\itemsep1pt\parskip0pt\parsep0pt
\item
  We get:
\end{itemize}

\begin{verbatim}
Matt, double 1.0=
d 1
Matt, char a=
c a
Matt, int 2=
i 2
\end{verbatim}

\subsubsection{Explicit Instantiation - part
3}\label{explicit-instantiation---part-3}

\begin{itemize}
\itemsep1pt\parskip0pt\parsep0pt
\item
  Explicit Instantiation:

  \begin{itemize}
  \itemsep1pt\parskip0pt\parsep0pt
  \item
    Forces instantiation of the function
  \item
    Must appear after the definition
  \item
    Must appear only once for given argument list
  \item
    Stops implicit instantiation
  \end{itemize}
\item
  So, mainly used by compiled library providers
\item
  Clients then \texttt{\#include} header and link to library
\end{itemize}

\begin{verbatim}
Linking CXX executable explicitInstantiationMain.x
Undefined symbols for architecture x86_64:
  "void f<float>(float)", referenced from:
\end{verbatim}

\subsubsection{Implicit Instantiation - part
1}\label{implicit-instantiation---part-1}

\begin{itemize}
\item
  Instantiated as they are used
\item
  Normally via \texttt{\#include} header files.
\item
  Given (library) header, that containts implementation:
\end{itemize}

\begin{Shaded}
\begin{Highlighting}[]
\OtherTok{#ifndef explicitInstantiation_h}
\OtherTok{#define explicitInstantiation_h}
\OtherTok{#include <iostream>}
\OtherTok{#include <typeinfo>}
\KeywordTok{template} \NormalTok{<}\KeywordTok{typename} \NormalTok{T> }\DataTypeTok{void} \NormalTok{f(T s) \{ std::cout << }\KeywordTok{typeid}\NormalTok{(T).name() << }\StringTok{" "} \NormalTok{<< s << }\CharTok{'\textbackslash{}n'}\NormalTok{; \}}
\OtherTok{#endif}
\end{Highlighting}
\end{Shaded}

\subsubsection{Implicit Instantiation - part
2}\label{implicit-instantiation---part-2}

\begin{itemize}
\itemsep1pt\parskip0pt\parsep0pt
\item
  Given client code:
\end{itemize}

\begin{Shaded}
\begin{Highlighting}[]
\OtherTok{#include <iostream>}
\OtherTok{#include "implicitInstantiation.h"}

\DataTypeTok{int} \NormalTok{main(}\DataTypeTok{int} \NormalTok{argc, }\DataTypeTok{char}\NormalTok{** argv)}
\NormalTok{\{}
  \NormalTok{std::cout << }\StringTok{"Matt, double 1.0="} \NormalTok{<< std::endl;}
  \NormalTok{f(}\FloatTok{1.0}\NormalTok{);}
  \NormalTok{std::cout << }\StringTok{"Matt, char a="} \NormalTok{<< std::endl;}
  \NormalTok{f('a');}
  \NormalTok{std::cout << }\StringTok{"Matt, int 2="} \NormalTok{<< std::endl;}
  \NormalTok{f(}\DecValTok{2}\NormalTok{);}
  \NormalTok{std::cout << }\StringTok{"Matt, float 3.0="} \NormalTok{<< std::endl;}
  \NormalTok{f<}\DataTypeTok{float}\NormalTok{>(}\KeywordTok{static_cast}\NormalTok{<}\DataTypeTok{float}\NormalTok{>(}\FloatTok{3.0}\NormalTok{));  }\CommentTok{// no compile error}
\NormalTok{\}}
\end{Highlighting}
\end{Shaded}

\begin{itemize}
\itemsep1pt\parskip0pt\parsep0pt
\item
  We get:
\end{itemize}

\begin{verbatim}
Matt, double 1.0=
d 1
Matt, char a=
c a
Matt, int 2=
i 2
Matt, float 3.0=
f 3
\end{verbatim}

\subsubsection{Homework 19}\label{homework-19-1}

\begin{itemize}
\itemsep1pt\parskip0pt\parsep0pt
\item
  Write a template function \texttt{AGreaterThanB} that compares two
  input agruments of type \texttt{T} and returns a \texttt{bool} if
  \texttt{A} is greater than \texttt{B}. The function should be able to
  handle either \texttt{int}, \texttt{float} or \texttt{string} entries
  (for the latter you will need to decide how to rank by size)
\item
  Try out the different types of explicit and implicit instantiation
\item
  Advanced/optional: write a template function that performs a binary
  search on the contents of a vector containing either \texttt{int},
  \texttt{float} or \texttt{string} entries (you could adapt this
  \href{http://www.cplusplus.com/reference/algorithm/find/}{equivalent
  to} ) code, N.B. the \texttt{vector} will need to be sorted by size
\end{itemize}

\subsection{Class Templates}\label{class-templates}

\subsubsection{Class Templates Example - part
1}\label{class-templates-example---part-1}

\begin{itemize}
\itemsep1pt\parskip0pt\parsep0pt
\item
  If you understand template functions, then template classes are easy!
\item
  Refering to
  \href{http://www.cplusplus.com/doc/tutorial/templates/}{this
  tutorial}, an example:
\end{itemize}

Header:

\begin{Shaded}
\begin{Highlighting}[]
\KeywordTok{template} \NormalTok{<}\KeywordTok{typename} \NormalTok{T> }\KeywordTok{class} \NormalTok{MyPair \{}
  \NormalTok{T m_Values[}\DecValTok{2}\NormalTok{];}

\KeywordTok{public}\NormalTok{:}
  \NormalTok{MyPair(}\DataTypeTok{const} \NormalTok{T &first, }\DataTypeTok{const} \NormalTok{T &second);}
  \NormalTok{T getMax() }\DataTypeTok{const}\NormalTok{;}
\NormalTok{\};}
\OtherTok{#include "pairClassExample.cc"}
\end{Highlighting}
\end{Shaded}

\subsubsection{Class Templates Example - part
2}\label{class-templates-example---part-2}

Implementation:

\begin{Shaded}
\begin{Highlighting}[]
\KeywordTok{template} \NormalTok{<}\KeywordTok{typename} \NormalTok{T>}
\NormalTok{MyPair<T>::MyPair(}\DataTypeTok{const} \NormalTok{T& first, }\DataTypeTok{const} \NormalTok{T& second)}
\NormalTok{\{}
  \NormalTok{m_Values[}\DecValTok{0}\NormalTok{] = first;}
  \NormalTok{m_Values[}\DecValTok{1}\NormalTok{] = second;}
\NormalTok{\}}

\KeywordTok{template} \NormalTok{<}\KeywordTok{typename} \NormalTok{T>}
\NormalTok{T}
\NormalTok{MyPair<T>::getMax() }\DataTypeTok{const}
\NormalTok{\{}
  \KeywordTok{if} \NormalTok{(m_Values[}\DecValTok{0}\NormalTok{] > m_Values[}\DecValTok{1}\NormalTok{])}
    \KeywordTok{return} \NormalTok{m_Values[}\DecValTok{0}\NormalTok{];}
  \KeywordTok{else}
    \KeywordTok{return} \NormalTok{m_Values[}\DecValTok{1}\NormalTok{];}
\NormalTok{\}}
\end{Highlighting}
\end{Shaded}

\subsubsection{Class Templates Example - part
3}\label{class-templates-example---part-3}

Usage:

\begin{Shaded}
\begin{Highlighting}[]
\OtherTok{#include "pairClassExample.h"}
\OtherTok{#include <iostream>}

\DataTypeTok{int} \NormalTok{main(}\DataTypeTok{int} \NormalTok{argc, }\DataTypeTok{char}\NormalTok{** argv)}
\NormalTok{\{}
  \NormalTok{MyPair<}\DataTypeTok{int}\NormalTok{> a(}\DecValTok{1}\NormalTok{,}\DecValTok{2}\NormalTok{);}
  \NormalTok{std::cout << }\StringTok{"Max is:"} \NormalTok{<< a.getMax() << std::endl;}
\NormalTok{\}}
\end{Highlighting}
\end{Shaded}

\subsubsection{Quick Comments}\label{quick-comments}

\begin{itemize}
\itemsep1pt\parskip0pt\parsep0pt
\item
  Implementation, 3 uses of parameter T
\item
  Same Implicit/Explicit instantiation rules
\item
  Note implicit requirements, eg. operator \textgreater{}

  \begin{itemize}
  \itemsep1pt\parskip0pt\parsep0pt
  \item
    Remember the 2 stage compilation
  \item
    Remember code not instantiated until its used
  \item
    Take Unit Testing Seriously!
  \end{itemize}
\end{itemize}

\subsubsection{Template Specialisation}\label{template-specialisation}

\begin{itemize}
\itemsep1pt\parskip0pt\parsep0pt
\item
  If template defined for type T
\item
  Full specialisation - special case for a specific type eg. char
\item
  Partial specialisation - special case for a type that still templates,
  e.g.~T*
\end{itemize}

\begin{verbatim}
template <typename T> class MyVector {
template <> class MyVector<char> {  // full specialisation
template <typename T> MyVector<T*> { // partial specialisation
\end{verbatim}

\subsubsection{Homework 20}\label{homework-20}

\begin{itemize}
\itemsep1pt\parskip0pt\parsep0pt
\item
  Implement the above class \texttt{MyPair} template
\item
  Try out with both Implicit and Explicit instantiation
\item
  Add a \texttt{Swap()} method that switches the contents of
  \texttt{m\_Values{[}0{]}} and \texttt{m\_Values{[}1{]}}
\end{itemize}

\subsection{Summary}\label{summary}

\subsubsection{Putting It Together -
Subsystems}\label{putting-it-together---subsystems}

\begin{itemize}
\itemsep1pt\parskip0pt\parsep0pt
\item
  Program to interfaces
\item
  Inject dependencies
\item
  Always make fully initialised object
\item
  Always use smart pointers
\item
  Use exceptions for errors
\item
  RAII very useful
\item
  Results in a more flexible, extensible, robust design
\end{itemize}

\subsubsection{Putting It Together -
OOP}\label{putting-it-together---oop}

\begin{itemize}
\itemsep1pt\parskip0pt\parsep0pt
\item
  Prefer composition over inheritance
\item
  Use constructional patterns to assemble code
\end{itemize}
